%% Generated by Sphinx.
\def\sphinxdocclass{report}
\documentclass[letterpaper,10pt,english]{sphinxmanual}
\ifdefined\pdfpxdimen
   \let\sphinxpxdimen\pdfpxdimen\else\newdimen\sphinxpxdimen
\fi \sphinxpxdimen=.75bp\relax

\PassOptionsToPackage{warn}{textcomp}
\usepackage[utf8]{inputenc}
\ifdefined\DeclareUnicodeCharacter
% support both utf8 and utf8x syntaxes
  \ifdefined\DeclareUnicodeCharacterAsOptional
    \def\sphinxDUC#1{\DeclareUnicodeCharacter{"#1}}
  \else
    \let\sphinxDUC\DeclareUnicodeCharacter
  \fi
  \sphinxDUC{00A0}{\nobreakspace}
  \sphinxDUC{2500}{\sphinxunichar{2500}}
  \sphinxDUC{2502}{\sphinxunichar{2502}}
  \sphinxDUC{2514}{\sphinxunichar{2514}}
  \sphinxDUC{251C}{\sphinxunichar{251C}}
  \sphinxDUC{2572}{\textbackslash}
\fi
\usepackage{cmap}
\usepackage[T1]{fontenc}
\usepackage{amsmath,amssymb,amstext}
\usepackage{babel}



\usepackage{times}
\expandafter\ifx\csname T@LGR\endcsname\relax
\else
% LGR was declared as font encoding
  \substitutefont{LGR}{\rmdefault}{cmr}
  \substitutefont{LGR}{\sfdefault}{cmss}
  \substitutefont{LGR}{\ttdefault}{cmtt}
\fi
\expandafter\ifx\csname T@X2\endcsname\relax
  \expandafter\ifx\csname T@T2A\endcsname\relax
  \else
  % T2A was declared as font encoding
    \substitutefont{T2A}{\rmdefault}{cmr}
    \substitutefont{T2A}{\sfdefault}{cmss}
    \substitutefont{T2A}{\ttdefault}{cmtt}
  \fi
\else
% X2 was declared as font encoding
  \substitutefont{X2}{\rmdefault}{cmr}
  \substitutefont{X2}{\sfdefault}{cmss}
  \substitutefont{X2}{\ttdefault}{cmtt}
\fi


\usepackage[Bjarne]{fncychap}
\usepackage{sphinx}

\fvset{fontsize=\small}
\usepackage{geometry}


% Include hyperref last.
\usepackage{hyperref}
% Fix anchor placement for figures with captions.
\usepackage{hypcap}% it must be loaded after hyperref.
% Set up styles of URL: it should be placed after hyperref.
\urlstyle{same}


\usepackage{sphinxmessages}
\setcounter{tocdepth}{1}



\title{scdc}
\date{Aug 31, 2021}
\release{0.0}
\author{Yonit Hochberg,
Eric David Kramer,
Noah Kurinsky,
and Benjamin V.\@{} Lehmann}
\newcommand{\sphinxlogo}{\vbox{}}
\renewcommand{\releasename}{Release}
\makeindex
\begin{document}

\pagestyle{empty}
\sphinxmaketitle
\pagestyle{plain}
\sphinxtableofcontents
\pagestyle{normal}
\phantomsection\label{\detokenize{index::doc}}


\sphinxtitleref{scdc} stands for \sphinxtitleref{superconductor down\sphinxhyphen{}conversion}. This code simulates the relaxation of quasiparticle excitations produced by an energy deposit in a superconductor, with particular attention to energy deposits from dark matter scattering. The code was developed to study directional correlations between the initial and final states, and is bundled with some simple tools to quantify that relationship.

The basic structure of the calculation is as follows: an energy deposit produces initial quasiparticles and/or phonons. The quasiparticles can relax by emitting phonons, and sufficiently energetic phonons can decay to quasiparticle pairs. This process continues until all quasiparticles are too low\sphinxhyphen{}energy to emit a phonon, and all phonons are too low\sphinxhyphen{}energy to produce a quasiparticle pair. These excitations constitute the final state, and we say that they are \sphinxtitleref{ballistic}. The full set of excitations connecting initial to final states is generated as a tree of python objects, so every step of the shower can be inspected directly.

The main non\sphinxhyphen{}triviality is that the kinematical equations governing phonons and quasiparticles cannot be solved analytically, so the code solves them numerically.


\chapter{Contents}
\label{\detokenize{index:contents}}

\section{Kinematics}
\label{\detokenize{kinematics:kinematics}}\label{\detokenize{kinematics::doc}}
We now review the kinematics of phonons and quasiparticles as treated by the code. Note that we work in the “clean” limit, where crystal momentum is conserved.


\subsection{Units}
\label{\detokenize{kinematics:units}}
In the code, we work in “material” units, setting \(v_\ell = k_F = \Delta = 1\). This implies that \(\hbar = 2\gamma\sqrt z\), where \(z\equiv E_F/\Delta\).


\subsection{Dispersion relations}
\label{\detokenize{kinematics:dispersion-relations}}
The phonon dispersion relation is \(E=\hbar q\). This is invertible: a phonon’s energy uniquely determines its momentum. The same is not true for a quasiparticle, which has dispersion relation
\begin{equation*}
\begin{split}E = \left[1 + \left(\frac{\hbar^2 k^2}{2m_e} - z\right)^2\right]^{1/2}.\end{split}
\end{equation*}
The inverse is requires a choice of sign:
\begin{equation*}
\begin{split}k = \frac{1}{\hbar}\left[
        2m_e\left(
            z \pm\sqrt{E^2 - 1}
        \right)
    \right]^{1/2}.\end{split}
\end{equation*}
This corresponds to the fact that the energy is minimized (\(E=\Delta\)) at the Fermi surface, \(k=k_F\), and increases in either direction. In the code, \sphinxstyleemphasis{if a quasiparticle is instantiated with only energy specified, we choose this sign randomly.}


\subsection{Phonon decay to quasiparticles}
\label{\detokenize{kinematics:phonon-decay-to-quasiparticles}}
A phonon can decay to quasiparticles if it has enough energy to produce a quasiparticle pair, i.e., \(E>2\Delta\). Such a decay is characterized by the energy \(\omega\) of one of the two quasiparticles (and the azimuthal angle, which is always uniform\sphinxhyphen{}random). We take the differential rate in this variable from eq. (27) of Kaplan et al. (1976):
\begin{equation*}
\begin{split}\frac{\mathrm d\Gamma}{\mathrm d\omega} =
    \frac{1}{\sqrt{\omega^2 - \Delta^2}}
    \frac{\omega\left(E-\omega\right) + \Delta^2}
         {\sqrt{(E-\omega)^2 - \Delta^2}}
.\end{split}
\end{equation*}
There is a complication here, however, because this gives the differential rate as a function of quasiparticle \sphinxstyleemphasis{energy.} As we just discussed, the energy does not uniquely determine the momentum of a quasiparticle, which is essential to ensure that we satisfy conservation of momentum. Since the decay rate into each allowed final\sphinxhyphen{}state momentum should be equivalent, we \sphinxstyleemphasis{randomly} select the sign in the inverse dispersion relation for each quasiparticle.

But this introduces yet another complication: it is possible that one or more of these sign combinations is not kinematically allowed, in that they can produce decays with \(|\cos\theta|>1\). Thus, we first select \(\omega\) according to the above distribution, and then check whether each sign combination is kinematically allowed. We randomly select from the allowed combinations. These are called \sphinxstyleemphasis{candidate} final states.


\subsection{Phonon emission by quasiparticles}
\label{\detokenize{kinematics:phonon-emission-by-quasiparticles}}

\section{Structure of the code}
\label{\detokenize{code_structure:structure-of-the-code}}\label{\detokenize{code_structure::doc}}

\subsection{Events and overall structure of the calculation}
\label{\detokenize{code_structure:module-scdc.event}}\label{\detokenize{code_structure:events-and-overall-structure-of-the-calculation}}\index{module@\spxentry{module}!scdc.event@\spxentry{scdc.event}}\index{scdc.event@\spxentry{scdc.event}!module@\spxentry{module}}
The structure of the calculation is as follows.

A scattering event produces two quasiparticles. Each of those quasiparticles
can subsequently relax by emitting phonons. If sufficiently energetic, those
phonons can themselves decay to quasiparticle pairs. The process continues
until none of the products have enough energy to undergo another scatter or
decay.

This picture describes a tree of particles with interactions as the nodes of
the tree, and we use a very similar structure to implement the calculation
programmatically. Our tree consists of three types of object:
\begin{itemize}
\item {} 
Event

\item {} 
Interaction

\item {} 
Particle

\end{itemize}

In tree language, {\hyperref[\detokenize{code_structure:scdc.event.Event}]{\sphinxcrossref{\sphinxcode{\sphinxupquote{Event}}}}} objects are nodes and \sphinxcode{\sphinxupquote{Particle}} objects
are edges. \sphinxcode{\sphinxupquote{Interaction}} objects couple a single {\hyperref[\detokenize{code_structure:scdc.event.Event}]{\sphinxcrossref{\sphinxcode{\sphinxupquote{Event}}}}} object to
a collection of \sphinxcode{\sphinxupquote{Particle}} objects.

Why not just have \sphinxcode{\sphinxupquote{Interaction}} and \sphinxcode{\sphinxupquote{Particle}} objects? Because a
third type ({\hyperref[\detokenize{code_structure:scdc.event.Event}]{\sphinxcrossref{\sphinxcode{\sphinxupquote{Event}}}}}) is required to specify the relationship between the
two. The \sphinxcode{\sphinxupquote{Interaction}} types certainly need to know
about \sphinxcode{\sphinxupquote{Particle}} types in order to create the final state of each
interaction. But then how would a \sphinxcode{\sphinxupquote{Particle}} know about
the \sphinxcode{\sphinxupquote{Interaction}} types for which it can be an initial state?
Alternatively, how would each \sphinxcode{\sphinxupquote{Interaction}} type know about all the
others? At a technical level, this would require a circular import. At a
conceptual level, it indicates a mutual dependency between \sphinxcode{\sphinxupquote{Particle}}
and \sphinxcode{\sphinxupquote{Interaction}} types that deserves its own container.

So instead, the structure is as follows. The nodes in the tree
are {\hyperref[\detokenize{code_structure:scdc.event.Event}]{\sphinxcrossref{\sphinxcode{\sphinxupquote{Event}}}}} objects. \sphinxcode{\sphinxupquote{Particle}} objects are the edges
between {\hyperref[\detokenize{code_structure:scdc.event.Event}]{\sphinxcrossref{\sphinxcode{\sphinxupquote{Event}}}}} objects. The {\hyperref[\detokenize{code_structure:scdc.event.Event}]{\sphinxcrossref{\sphinxcode{\sphinxupquote{Event}}}}} objects “know”
what \sphinxcode{\sphinxupquote{Interaction}} types are possible for a given \sphinxcode{\sphinxupquote{Particle}}, and
they use such an \sphinxcode{\sphinxupquote{Interaction}} object internally to produce a new final
state. The {\hyperref[\detokenize{code_structure:scdc.event.Event}]{\sphinxcrossref{\sphinxcode{\sphinxupquote{Event}}}}} can then create new {\hyperref[\detokenize{code_structure:scdc.event.Event}]{\sphinxcrossref{\sphinxcode{\sphinxupquote{Event}}}}} nodes at which to
terminate each final\sphinxhyphen{}state \sphinxcode{\sphinxupquote{Particle}}.

Each \sphinxcode{\sphinxupquote{Interaction}} can also be physically prohibited. When this happens,
the “initial” state of such an \sphinxcode{\sphinxupquote{Interaction}} is a leaf of the tree.
\index{Event (class in scdc.event)@\spxentry{Event}\spxextra{class in scdc.event}}

\begin{fulllineitems}
\phantomsection\label{\detokenize{code_structure:scdc.event.Event}}\pysiglinewithargsret{\sphinxbfcode{\sphinxupquote{class }}\sphinxcode{\sphinxupquote{scdc.event.}}\sphinxbfcode{\sphinxupquote{Event}}}{\emph{\DUrole{n}{initial\_state}}, \emph{\DUrole{n}{material}}, \emph{\DUrole{o}{*}\DUrole{n}{args}}, \emph{\DUrole{o}{**}\DUrole{n}{kwargs}}}{}
Interaction tree node.
\begin{quote}\begin{description}
\item[{Parameters}] \leavevmode\begin{itemize}
\item {} 
\sphinxstyleliteralstrong{\sphinxupquote{initial\_state}} (\sphinxcode{\sphinxupquote{list}} of \sphinxcode{\sphinxupquote{Particle}}) \textendash{} particles in the
initial state. The initial state is always listlike, even if it
contains only one object.

\item {} 
\sphinxstyleliteralstrong{\sphinxupquote{material}} (\sphinxcode{\sphinxupquote{Material}}) \textendash{} the material in which this takes place.

\item {} 
\sphinxstyleliteralstrong{\sphinxupquote{final\_state}} (\sphinxcode{\sphinxupquote{list}} of \sphinxcode{\sphinxupquote{Particle}}, optional) \textendash{} particles in
the final state. This is passed to the internal \sphinxcode{\sphinxupquote{Interaction}}
object if supplied.

\end{itemize}

\end{description}\end{quote}
\index{initial\_state (scdc.event.Event attribute)@\spxentry{initial\_state}\spxextra{scdc.event.Event attribute}}

\begin{fulllineitems}
\phantomsection\label{\detokenize{code_structure:scdc.event.Event.initial_state}}\pysigline{\sphinxbfcode{\sphinxupquote{initial\_state}}}
particles in the
initial state. The initial state is always a list, even if it
contains only one object.
\begin{quote}\begin{description}
\item[{Type}] \leavevmode
\sphinxcode{\sphinxupquote{list}} of \sphinxcode{\sphinxupquote{Particle}}

\end{description}\end{quote}

\end{fulllineitems}

\index{final\_state (scdc.event.Event attribute)@\spxentry{final\_state}\spxextra{scdc.event.Event attribute}}

\begin{fulllineitems}
\phantomsection\label{\detokenize{code_structure:scdc.event.Event.final_state}}\pysigline{\sphinxbfcode{\sphinxupquote{final\_state}}}
particles in
the final state if and only if supplied at initialization.
\begin{quote}\begin{description}
\item[{Type}] \leavevmode
\sphinxcode{\sphinxupquote{list}} of \sphinxcode{\sphinxupquote{Particle}}, optional

\end{description}\end{quote}

\end{fulllineitems}



\begin{fulllineitems}
\pysigline{\sphinxbfcode{\sphinxupquote{material~(obj}}}
\sphinxtitleref{Material}): the material in which this takes place.

\end{fulllineitems}

\index{interaction (scdc.event.Event attribute)@\spxentry{interaction}\spxextra{scdc.event.Event attribute}}

\begin{fulllineitems}
\phantomsection\label{\detokenize{code_structure:scdc.event.Event.interaction}}\pysigline{\sphinxbfcode{\sphinxupquote{interaction}}}
the interaction which can take place
given the supplied initial state.
\begin{quote}\begin{description}
\item[{Type}] \leavevmode
\sphinxcode{\sphinxupquote{Interaction}}

\end{description}\end{quote}

\end{fulllineitems}

\index{out (scdc.event.Event attribute)@\spxentry{out}\spxextra{scdc.event.Event attribute}}

\begin{fulllineitems}
\phantomsection\label{\detokenize{code_structure:scdc.event.Event.out}}\pysigline{\sphinxbfcode{\sphinxupquote{out}}}
particle \sphinxtitleref{Event} objects.
\begin{quote}\begin{description}
\item[{Type}] \leavevmode
\sphinxcode{\sphinxupquote{list}} of \sphinxcode{\sphinxupquote{Particle}}

\end{description}\end{quote}

\end{fulllineitems}

\index{final (scdc.event.Event attribute)@\spxentry{final}\spxextra{scdc.event.Event attribute}}

\begin{fulllineitems}
\phantomsection\label{\detokenize{code_structure:scdc.event.Event.final}}\pysigline{\sphinxbfcode{\sphinxupquote{final}}}
\sphinxtitleref{True} if no further interaction can take place.
\begin{quote}\begin{description}
\item[{Type}] \leavevmode
bool

\end{description}\end{quote}

\end{fulllineitems}

\index{act() (scdc.event.Event method)@\spxentry{act()}\spxextra{scdc.event.Event method}}

\begin{fulllineitems}
\phantomsection\label{\detokenize{code_structure:scdc.event.Event.act}}\pysiglinewithargsret{\sphinxbfcode{\sphinxupquote{act}}}{}{}
Run the interaction and generate a final state.

\end{fulllineitems}

\index{chain() (scdc.event.Event method)@\spxentry{chain()}\spxextra{scdc.event.Event method}}

\begin{fulllineitems}
\phantomsection\label{\detokenize{code_structure:scdc.event.Event.chain}}\pysiglinewithargsret{\sphinxbfcode{\sphinxupquote{chain}}}{}{}
Run the interaction and run particle interactions recursively.

\end{fulllineitems}

\index{leaf\_events() (scdc.event.Event property)@\spxentry{leaf\_events()}\spxextra{scdc.event.Event property}}

\begin{fulllineitems}
\phantomsection\label{\detokenize{code_structure:scdc.event.Event.leaf_events}}\pysigline{\sphinxbfcode{\sphinxupquote{property }}\sphinxbfcode{\sphinxupquote{leaf\_events}}}
final events in the detector.

These are the leaves of the event tree, corresponding to the final
states that will be measured.
\begin{quote}\begin{description}
\item[{Type}] \leavevmode
\sphinxcode{\sphinxupquote{list}} of {\hyperref[\detokenize{code_structure:scdc.event.Event}]{\sphinxcrossref{\sphinxcode{\sphinxupquote{Event}}}}}

\end{description}\end{quote}

\end{fulllineitems}

\index{leaf\_particles() (scdc.event.Event property)@\spxentry{leaf\_particles()}\spxextra{scdc.event.Event property}}

\begin{fulllineitems}
\phantomsection\label{\detokenize{code_structure:scdc.event.Event.leaf_particles}}\pysigline{\sphinxbfcode{\sphinxupquote{property }}\sphinxbfcode{\sphinxupquote{leaf\_particles}}}
final particles after relaxation.
\begin{quote}\begin{description}
\item[{Type}] \leavevmode
\sphinxcode{\sphinxupquote{list}} of \sphinxcode{\sphinxupquote{Particle}}

\end{description}\end{quote}

\end{fulllineitems}


\end{fulllineitems}



\subsection{Particle objects}
\label{\detokenize{code_structure:module-scdc.particle}}\label{\detokenize{code_structure:particle-objects}}\index{module@\spxentry{module}!scdc.particle@\spxentry{scdc.particle}}\index{scdc.particle@\spxentry{scdc.particle}!module@\spxentry{module}}
This module defines the different types of (quasi)particle.
\index{DarkMatter (class in scdc.particle)@\spxentry{DarkMatter}\spxextra{class in scdc.particle}}

\begin{fulllineitems}
\phantomsection\label{\detokenize{code_structure:scdc.particle.DarkMatter}}\pysiglinewithargsret{\sphinxbfcode{\sphinxupquote{class }}\sphinxcode{\sphinxupquote{scdc.particle.}}\sphinxbfcode{\sphinxupquote{DarkMatter}}}{\emph{\DUrole{o}{**}\DUrole{n}{kwargs}}}{}
A dark matter particle.
\index{dispersion() (scdc.particle.DarkMatter method)@\spxentry{dispersion()}\spxextra{scdc.particle.DarkMatter method}}

\begin{fulllineitems}
\phantomsection\label{\detokenize{code_structure:scdc.particle.DarkMatter.dispersion}}\pysiglinewithargsret{\sphinxbfcode{\sphinxupquote{dispersion}}}{\emph{\DUrole{n}{p}}}{}
Dispersion relation for dark matter.

Be careful: there is a need for conversion to the material’s own
canonical units. (NOT currently implemented.)
\begin{quote}\begin{description}
\item[{Parameters}] \leavevmode
\sphinxstyleliteralstrong{\sphinxupquote{p}} (\sphinxstyleliteralemphasis{\sphinxupquote{float}}) \textendash{} momentum in who\sphinxhyphen{}knows\sphinxhyphen{}what units…to fix.

\item[{Returns}] \leavevmode
dark matter energy in units of Delta.

\item[{Return type}] \leavevmode
float

\end{description}\end{quote}

\end{fulllineitems}


\end{fulllineitems}

\index{Particle (class in scdc.particle)@\spxentry{Particle}\spxextra{class in scdc.particle}}

\begin{fulllineitems}
\phantomsection\label{\detokenize{code_structure:scdc.particle.Particle}}\pysiglinewithargsret{\sphinxbfcode{\sphinxupquote{class }}\sphinxcode{\sphinxupquote{scdc.particle.}}\sphinxbfcode{\sphinxupquote{Particle}}}{\emph{\DUrole{o}{**}\DUrole{n}{kwargs}}}{}
Represents a single phonon or quasiparticle with a tree of children.

\sphinxtitleref{momentum} must be specified unless \sphinxtitleref{dispersion\_inverse} is defined, in
which case either \sphinxtitleref{momentum} or \sphinxtitleref{energy} can be specified.

All quantities are specified in material units.
\begin{quote}\begin{description}
\item[{Parameters}] \leavevmode\begin{itemize}
\item {} 
\sphinxstyleliteralstrong{\sphinxupquote{energy}} (\sphinxstyleliteralemphasis{\sphinxupquote{float}}\sphinxstyleliteralemphasis{\sphinxupquote{, }}\sphinxstyleliteralemphasis{\sphinxupquote{optional}}) \textendash{} energy in units of Delta.

\item {} 
\sphinxstyleliteralstrong{\sphinxupquote{momentum}} (\sphinxstyleliteralemphasis{\sphinxupquote{float}}\sphinxstyleliteralemphasis{\sphinxupquote{, }}\sphinxstyleliteralemphasis{\sphinxupquote{optional}}) \textendash{} momentum in canonical units.

\item {} 
\sphinxstyleliteralstrong{\sphinxupquote{cos\_theta}} (\sphinxstyleliteralemphasis{\sphinxupquote{float}}) \textendash{} cosine of angle wrt global z.

\item {} 
\sphinxstyleliteralstrong{\sphinxupquote{material}} (\sphinxcode{\sphinxupquote{Material}}) \textendash{} material properties.

\item {} 
\sphinxstyleliteralstrong{\sphinxupquote{pid}} (\sphinxstyleliteralemphasis{\sphinxupquote{object}}\sphinxstyleliteralemphasis{\sphinxupquote{, }}\sphinxstyleliteralemphasis{\sphinxupquote{optional}}) \textendash{} optional ID label. Defaults to \sphinxcode{\sphinxupquote{None}}.

\item {} 
\sphinxstyleliteralstrong{\sphinxupquote{tolerance}} (\sphinxstyleliteralemphasis{\sphinxupquote{float}}\sphinxstyleliteralemphasis{\sphinxupquote{, }}\sphinxstyleliteralemphasis{\sphinxupquote{optional}}) \textendash{} tolerance for \(\cos\theta > 1\).
Defaults to \sphinxcode{\sphinxupquote{1e\sphinxhyphen{}2}}.

\end{itemize}

\end{description}\end{quote}
\index{energy (scdc.particle.Particle attribute)@\spxentry{energy}\spxextra{scdc.particle.Particle attribute}}

\begin{fulllineitems}
\phantomsection\label{\detokenize{code_structure:scdc.particle.Particle.energy}}\pysigline{\sphinxbfcode{\sphinxupquote{energy}}}
energy in units of Delta.
\begin{quote}\begin{description}
\item[{Type}] \leavevmode
float

\end{description}\end{quote}

\end{fulllineitems}

\index{momentum (scdc.particle.Particle attribute)@\spxentry{momentum}\spxextra{scdc.particle.Particle attribute}}

\begin{fulllineitems}
\phantomsection\label{\detokenize{code_structure:scdc.particle.Particle.momentum}}\pysigline{\sphinxbfcode{\sphinxupquote{momentum}}}
momentum in canonical units.
\begin{quote}\begin{description}
\item[{Type}] \leavevmode
float, optional

\end{description}\end{quote}

\end{fulllineitems}

\index{cos\_theta (scdc.particle.Particle attribute)@\spxentry{cos\_theta}\spxextra{scdc.particle.Particle attribute}}

\begin{fulllineitems}
\phantomsection\label{\detokenize{code_structure:scdc.particle.Particle.cos_theta}}\pysigline{\sphinxbfcode{\sphinxupquote{cos\_theta}}}
cosine of angle wrt global z.
\begin{quote}\begin{description}
\item[{Type}] \leavevmode
float

\end{description}\end{quote}

\end{fulllineitems}

\index{material (scdc.particle.Particle attribute)@\spxentry{material}\spxextra{scdc.particle.Particle attribute}}

\begin{fulllineitems}
\phantomsection\label{\detokenize{code_structure:scdc.particle.Particle.material}}\pysigline{\sphinxbfcode{\sphinxupquote{material}}}
material properties.
\begin{quote}\begin{description}
\item[{Type}] \leavevmode
\sphinxcode{\sphinxupquote{Material}}

\end{description}\end{quote}

\end{fulllineitems}

\index{pid (scdc.particle.Particle attribute)@\spxentry{pid}\spxextra{scdc.particle.Particle attribute}}

\begin{fulllineitems}
\phantomsection\label{\detokenize{code_structure:scdc.particle.Particle.pid}}\pysigline{\sphinxbfcode{\sphinxupquote{pid}}}
optional ID label.
\begin{quote}\begin{description}
\item[{Type}] \leavevmode
object, optional

\end{description}\end{quote}

\end{fulllineitems}

\index{tolerance (scdc.particle.Particle attribute)@\spxentry{tolerance}\spxextra{scdc.particle.Particle attribute}}

\begin{fulllineitems}
\phantomsection\label{\detokenize{code_structure:scdc.particle.Particle.tolerance}}\pysigline{\sphinxbfcode{\sphinxupquote{tolerance}}}
tolerance for \(\cos\theta > 1\).
\begin{quote}\begin{description}
\item[{Type}] \leavevmode
float, optional

\end{description}\end{quote}

\end{fulllineitems}

\index{parent (scdc.particle.Particle attribute)@\spxentry{parent}\spxextra{scdc.particle.Particle attribute}}

\begin{fulllineitems}
\phantomsection\label{\detokenize{code_structure:scdc.particle.Particle.parent}}\pysigline{\sphinxbfcode{\sphinxupquote{parent}}}
parent particle.
\begin{quote}\begin{description}
\item[{Type}] \leavevmode
{\hyperref[\detokenize{code_structure:scdc.particle.Particle}]{\sphinxcrossref{\sphinxcode{\sphinxupquote{Particle}}}}}

\end{description}\end{quote}

\end{fulllineitems}

\index{origin (scdc.particle.Particle attribute)@\spxentry{origin}\spxextra{scdc.particle.Particle attribute}}

\begin{fulllineitems}
\phantomsection\label{\detokenize{code_structure:scdc.particle.Particle.origin}}\pysigline{\sphinxbfcode{\sphinxupquote{origin}}}
the originating \sphinxtitleref{Event} for this particle.
\begin{quote}\begin{description}
\item[{Type}] \leavevmode
\sphinxcode{\sphinxupquote{Event}}

\end{description}\end{quote}

\end{fulllineitems}

\index{dest (scdc.particle.Particle attribute)@\spxentry{dest}\spxextra{scdc.particle.Particle attribute}}

\begin{fulllineitems}
\phantomsection\label{\detokenize{code_structure:scdc.particle.Particle.dest}}\pysigline{\sphinxbfcode{\sphinxupquote{dest}}}
the destination \sphinxtitleref{Event} for this particle.
\begin{quote}\begin{description}
\item[{Type}] \leavevmode
\sphinxcode{\sphinxupquote{Event}}

\end{description}\end{quote}

\end{fulllineitems}

\index{shortname (scdc.particle.Particle attribute)@\spxentry{shortname}\spxextra{scdc.particle.Particle attribute}}

\begin{fulllineitems}
\phantomsection\label{\detokenize{code_structure:scdc.particle.Particle.shortname}}\pysigline{\sphinxbfcode{\sphinxupquote{shortname}}}
a short label for this particle type.
\begin{quote}\begin{description}
\item[{Type}] \leavevmode
str

\end{description}\end{quote}

\end{fulllineitems}

\index{dispersion() (scdc.particle.Particle method)@\spxentry{dispersion()}\spxextra{scdc.particle.Particle method}}

\begin{fulllineitems}
\phantomsection\label{\detokenize{code_structure:scdc.particle.Particle.dispersion}}\pysiglinewithargsret{\sphinxbfcode{\sphinxupquote{dispersion}}}{\emph{\DUrole{n}{k}}}{}
Disperion relation: wavenumber to energy.

\end{fulllineitems}

\index{dispersion\_inverse() (scdc.particle.Particle method)@\spxentry{dispersion\_inverse()}\spxextra{scdc.particle.Particle method}}

\begin{fulllineitems}
\phantomsection\label{\detokenize{code_structure:scdc.particle.Particle.dispersion_inverse}}\pysiglinewithargsret{\sphinxbfcode{\sphinxupquote{dispersion\_inverse}}}{\emph{\DUrole{n}{E}}}{}
Inverse dispersion relation: energy to wavenumber.

\end{fulllineitems}

\index{energy() (scdc.particle.Particle property)@\spxentry{energy()}\spxextra{scdc.particle.Particle property}}

\begin{fulllineitems}
\phantomsection\label{\detokenize{code_structure:id0}}\pysigline{\sphinxbfcode{\sphinxupquote{property }}\sphinxbfcode{\sphinxupquote{energy}}}
energy in units of Delta.
\begin{quote}\begin{description}
\item[{Type}] \leavevmode
float

\end{description}\end{quote}

\end{fulllineitems}

\index{momentum() (scdc.particle.Particle property)@\spxentry{momentum()}\spxextra{scdc.particle.Particle property}}

\begin{fulllineitems}
\phantomsection\label{\detokenize{code_structure:id1}}\pysigline{\sphinxbfcode{\sphinxupquote{property }}\sphinxbfcode{\sphinxupquote{momentum}}}
momentum in material units.
\begin{quote}\begin{description}
\item[{Type}] \leavevmode
float

\end{description}\end{quote}

\end{fulllineitems}


\end{fulllineitems}

\index{ParticleCollection (class in scdc.particle)@\spxentry{ParticleCollection}\spxextra{class in scdc.particle}}

\begin{fulllineitems}
\phantomsection\label{\detokenize{code_structure:scdc.particle.ParticleCollection}}\pysiglinewithargsret{\sphinxbfcode{\sphinxupquote{class }}\sphinxcode{\sphinxupquote{scdc.particle.}}\sphinxbfcode{\sphinxupquote{ParticleCollection}}}{\emph{\DUrole{o}{*}\DUrole{n}{args}}, \emph{\DUrole{o}{**}\DUrole{n}{kwargs}}}{}
Represents a collection of particles.
\begin{quote}\begin{description}
\item[{Parameters}] \leavevmode
\sphinxstyleliteralstrong{\sphinxupquote{particles}} (\sphinxcode{\sphinxupquote{list}} of {\hyperref[\detokenize{code_structure:scdc.particle.Particle}]{\sphinxcrossref{\sphinxcode{\sphinxupquote{Particle}}}}}) \textendash{} particles.

\end{description}\end{quote}
\index{particles (scdc.particle.ParticleCollection attribute)@\spxentry{particles}\spxextra{scdc.particle.ParticleCollection attribute}}

\begin{fulllineitems}
\phantomsection\label{\detokenize{code_structure:scdc.particle.ParticleCollection.particles}}\pysigline{\sphinxbfcode{\sphinxupquote{particles}}}
particles.
\begin{quote}\begin{description}
\item[{Type}] \leavevmode
\sphinxcode{\sphinxupquote{list}} of {\hyperref[\detokenize{code_structure:scdc.particle.Particle}]{\sphinxcrossref{\sphinxcode{\sphinxupquote{Particle}}}}}

\end{description}\end{quote}

\end{fulllineitems}

\index{energy (scdc.particle.ParticleCollection attribute)@\spxentry{energy}\spxextra{scdc.particle.ParticleCollection attribute}}

\begin{fulllineitems}
\phantomsection\label{\detokenize{code_structure:scdc.particle.ParticleCollection.energy}}\pysigline{\sphinxbfcode{\sphinxupquote{energy}}}
energy of each particle.
\begin{quote}\begin{description}
\item[{Type}] \leavevmode
\sphinxcode{\sphinxupquote{ndarray}} of float

\end{description}\end{quote}

\end{fulllineitems}

\index{momentum (scdc.particle.ParticleCollection attribute)@\spxentry{momentum}\spxextra{scdc.particle.ParticleCollection attribute}}

\begin{fulllineitems}
\phantomsection\label{\detokenize{code_structure:scdc.particle.ParticleCollection.momentum}}\pysigline{\sphinxbfcode{\sphinxupquote{momentum}}}
momentum of each particle.
\begin{quote}\begin{description}
\item[{Type}] \leavevmode
\sphinxcode{\sphinxupquote{ndarray}} of float

\end{description}\end{quote}

\end{fulllineitems}

\index{cos\_theta (scdc.particle.ParticleCollection attribute)@\spxentry{cos\_theta}\spxextra{scdc.particle.ParticleCollection attribute}}

\begin{fulllineitems}
\phantomsection\label{\detokenize{code_structure:scdc.particle.ParticleCollection.cos_theta}}\pysigline{\sphinxbfcode{\sphinxupquote{cos\_theta}}}
cos\_theta of each particle.
\begin{quote}\begin{description}
\item[{Type}] \leavevmode
\sphinxcode{\sphinxupquote{ndarray}} of float

\end{description}\end{quote}

\end{fulllineitems}

\index{cos\_theta() (scdc.particle.ParticleCollection property)@\spxentry{cos\_theta()}\spxextra{scdc.particle.ParticleCollection property}}

\begin{fulllineitems}
\phantomsection\label{\detokenize{code_structure:id2}}\pysigline{\sphinxbfcode{\sphinxupquote{property }}\sphinxbfcode{\sphinxupquote{cos\_theta}}}
array of particle \sphinxtitleref{cos\_theta} values.
\begin{quote}\begin{description}
\item[{Type}] \leavevmode
\sphinxcode{\sphinxupquote{ndarray}}

\end{description}\end{quote}

\end{fulllineitems}

\index{dm() (scdc.particle.ParticleCollection property)@\spxentry{dm()}\spxextra{scdc.particle.ParticleCollection property}}

\begin{fulllineitems}
\phantomsection\label{\detokenize{code_structure:scdc.particle.ParticleCollection.dm}}\pysigline{\sphinxbfcode{\sphinxupquote{property }}\sphinxbfcode{\sphinxupquote{dm}}}
subcollection of DM particles.
\begin{quote}\begin{description}
\item[{Type}] \leavevmode
{\hyperref[\detokenize{code_structure:scdc.particle.ParticleCollection}]{\sphinxcrossref{\sphinxcode{\sphinxupquote{ParticleCollection}}}}}

\end{description}\end{quote}

\end{fulllineitems}

\index{energy() (scdc.particle.ParticleCollection property)@\spxentry{energy()}\spxextra{scdc.particle.ParticleCollection property}}

\begin{fulllineitems}
\phantomsection\label{\detokenize{code_structure:id3}}\pysigline{\sphinxbfcode{\sphinxupquote{property }}\sphinxbfcode{\sphinxupquote{energy}}}
array of particle energies.
\begin{quote}\begin{description}
\item[{Type}] \leavevmode
\sphinxcode{\sphinxupquote{ndarray}}

\end{description}\end{quote}

\end{fulllineitems}

\index{from\_npy() (scdc.particle.ParticleCollection class method)@\spxentry{from\_npy()}\spxextra{scdc.particle.ParticleCollection class method}}

\begin{fulllineitems}
\phantomsection\label{\detokenize{code_structure:scdc.particle.ParticleCollection.from_npy}}\pysiglinewithargsret{\sphinxbfcode{\sphinxupquote{classmethod }}\sphinxbfcode{\sphinxupquote{from\_npy}}}{\emph{\DUrole{n}{npy}}, \emph{\DUrole{n}{material}}}{}
Construct a \sphinxtitleref{ParticleCollection} from the numpy representation.
\begin{quote}\begin{description}
\item[{Parameters}] \leavevmode\begin{itemize}
\item {} 
\sphinxstyleliteralstrong{\sphinxupquote{npy}} (\sphinxcode{\sphinxupquote{ndarray}}) \textendash{} a structured array with columns as specified
in the \sphinxtitleref{to\_npy} method.

\item {} 
\sphinxstyleliteralstrong{\sphinxupquote{material}} (\sphinxcode{\sphinxupquote{Material}}) \textendash{} the material to associate with these
particles. The material is not stored in the numpy format.

\end{itemize}

\item[{Returns}] \leavevmode
a corresponding \sphinxtitleref{ParticleCollection}.

\item[{Return type}] \leavevmode
{\hyperref[\detokenize{code_structure:scdc.particle.ParticleCollection}]{\sphinxcrossref{\sphinxcode{\sphinxupquote{ParticleCollection}}}}}

\end{description}\end{quote}

\end{fulllineitems}

\index{momentum() (scdc.particle.ParticleCollection property)@\spxentry{momentum()}\spxextra{scdc.particle.ParticleCollection property}}

\begin{fulllineitems}
\phantomsection\label{\detokenize{code_structure:id4}}\pysigline{\sphinxbfcode{\sphinxupquote{property }}\sphinxbfcode{\sphinxupquote{momentum}}}
array of particle energies.
\begin{quote}\begin{description}
\item[{Type}] \leavevmode
\sphinxcode{\sphinxupquote{ndarray}}

\end{description}\end{quote}

\end{fulllineitems}

\index{nondark() (scdc.particle.ParticleCollection property)@\spxentry{nondark()}\spxextra{scdc.particle.ParticleCollection property}}

\begin{fulllineitems}
\phantomsection\label{\detokenize{code_structure:scdc.particle.ParticleCollection.nondark}}\pysigline{\sphinxbfcode{\sphinxupquote{property }}\sphinxbfcode{\sphinxupquote{nondark}}}
subcollection of non\sphinxhyphen{}DM particles.
\begin{quote}\begin{description}
\item[{Type}] \leavevmode
{\hyperref[\detokenize{code_structure:scdc.particle.ParticleCollection}]{\sphinxcrossref{\sphinxcode{\sphinxupquote{ParticleCollection}}}}}

\end{description}\end{quote}

\end{fulllineitems}

\index{phonons() (scdc.particle.ParticleCollection property)@\spxentry{phonons()}\spxextra{scdc.particle.ParticleCollection property}}

\begin{fulllineitems}
\phantomsection\label{\detokenize{code_structure:scdc.particle.ParticleCollection.phonons}}\pysigline{\sphinxbfcode{\sphinxupquote{property }}\sphinxbfcode{\sphinxupquote{phonons}}}
subcollection of phonons.
\begin{quote}\begin{description}
\item[{Type}] \leavevmode
{\hyperref[\detokenize{code_structure:scdc.particle.ParticleCollection}]{\sphinxcrossref{\sphinxcode{\sphinxupquote{ParticleCollection}}}}}

\end{description}\end{quote}

\end{fulllineitems}

\index{quasiparticles() (scdc.particle.ParticleCollection property)@\spxentry{quasiparticles()}\spxextra{scdc.particle.ParticleCollection property}}

\begin{fulllineitems}
\phantomsection\label{\detokenize{code_structure:scdc.particle.ParticleCollection.quasiparticles}}\pysigline{\sphinxbfcode{\sphinxupquote{property }}\sphinxbfcode{\sphinxupquote{quasiparticles}}}
subcollection of quasiparticles.
\begin{quote}\begin{description}
\item[{Type}] \leavevmode
{\hyperref[\detokenize{code_structure:scdc.particle.ParticleCollection}]{\sphinxcrossref{\sphinxcode{\sphinxupquote{ParticleCollection}}}}}

\end{description}\end{quote}

\end{fulllineitems}

\index{select() (scdc.particle.ParticleCollection method)@\spxentry{select()}\spxextra{scdc.particle.ParticleCollection method}}

\begin{fulllineitems}
\phantomsection\label{\detokenize{code_structure:scdc.particle.ParticleCollection.select}}\pysiglinewithargsret{\sphinxbfcode{\sphinxupquote{select}}}{\emph{\DUrole{n}{test}}}{}
Select a subset of particles which satisfy a test.

The argument \sphinxtitleref{test} is a function with call signature
\begin{quote}

\sphinxtitleref{test(parent, particle)}
\end{quote}

which returns \sphinxtitleref{True} if the leaf \sphinxtitleref{particle} in the chain arising from
the initial particle \sphinxtitleref{parent} satisifes the test.
\begin{quote}\begin{description}
\item[{Parameters}] \leavevmode
\sphinxstyleliteralstrong{\sphinxupquote{test}} (\sphinxstyleliteralemphasis{\sphinxupquote{function}}) \textendash{} the test function.

\item[{Returns}] \leavevmode
\begin{description}
\item[{collection consisting of particles which}] \leavevmode
satisfy the test.

\end{description}


\item[{Return type}] \leavevmode
{\hyperref[\detokenize{code_structure:scdc.particle.ParticleCollection}]{\sphinxcrossref{\sphinxcode{\sphinxupquote{ParticleCollection}}}}}

\end{description}\end{quote}

\end{fulllineitems}

\index{to\_npy() (scdc.particle.ParticleCollection method)@\spxentry{to\_npy()}\spxextra{scdc.particle.ParticleCollection method}}

\begin{fulllineitems}
\phantomsection\label{\detokenize{code_structure:scdc.particle.ParticleCollection.to_npy}}\pysiglinewithargsret{\sphinxbfcode{\sphinxupquote{to\_npy}}}{}{}
Represent this particle collection in numpy format for export.

The numpy format is a structured array with the following columns:
\begin{quote}

shortname    momentum    cos\_theta
\end{quote}

The numpy form REQUIRES momenta for all particles.
\begin{quote}\begin{description}
\item[{Returns}] \leavevmode
the array form of this collection.

\item[{Return type}] \leavevmode
\sphinxcode{\sphinxupquote{ndarray}}

\end{description}\end{quote}

\end{fulllineitems}


\end{fulllineitems}

\index{Phonon (class in scdc.particle)@\spxentry{Phonon}\spxextra{class in scdc.particle}}

\begin{fulllineitems}
\phantomsection\label{\detokenize{code_structure:scdc.particle.Phonon}}\pysiglinewithargsret{\sphinxbfcode{\sphinxupquote{class }}\sphinxcode{\sphinxupquote{scdc.particle.}}\sphinxbfcode{\sphinxupquote{Phonon}}}{\emph{\DUrole{o}{**}\DUrole{n}{kwargs}}}{}
A phonon.
\index{dispersion() (scdc.particle.Phonon method)@\spxentry{dispersion()}\spxextra{scdc.particle.Phonon method}}

\begin{fulllineitems}
\phantomsection\label{\detokenize{code_structure:scdc.particle.Phonon.dispersion}}\pysiglinewithargsret{\sphinxbfcode{\sphinxupquote{dispersion}}}{\emph{\DUrole{n}{k}}}{}
Dispersion relation for phonons.
\begin{quote}\begin{description}
\item[{Parameters}] \leavevmode
\sphinxstyleliteralstrong{\sphinxupquote{k}} (\sphinxstyleliteralemphasis{\sphinxupquote{float}}) \textendash{} wavenumber in material units.

\item[{Returns}] \leavevmode
phonon energy in material units.

\item[{Return type}] \leavevmode
float

\end{description}\end{quote}

\end{fulllineitems}

\index{dispersion\_inverse() (scdc.particle.Phonon method)@\spxentry{dispersion\_inverse()}\spxextra{scdc.particle.Phonon method}}

\begin{fulllineitems}
\phantomsection\label{\detokenize{code_structure:scdc.particle.Phonon.dispersion_inverse}}\pysiglinewithargsret{\sphinxbfcode{\sphinxupquote{dispersion\_inverse}}}{\emph{\DUrole{n}{E}}}{}
Inverse dispersion relation for phonons.
\begin{quote}\begin{description}
\item[{Parameters}] \leavevmode
\sphinxstyleliteralstrong{\sphinxupquote{E}} (\sphinxstyleliteralemphasis{\sphinxupquote{float}}) \textendash{} phonon energy in material units.

\item[{Returns}] \leavevmode
phonon wavenumber in material units.

\item[{Return type}] \leavevmode
float

\end{description}\end{quote}

\end{fulllineitems}


\end{fulllineitems}

\index{Quasiparticle (class in scdc.particle)@\spxentry{Quasiparticle}\spxextra{class in scdc.particle}}

\begin{fulllineitems}
\phantomsection\label{\detokenize{code_structure:scdc.particle.Quasiparticle}}\pysiglinewithargsret{\sphinxbfcode{\sphinxupquote{class }}\sphinxcode{\sphinxupquote{scdc.particle.}}\sphinxbfcode{\sphinxupquote{Quasiparticle}}}{\emph{\DUrole{o}{**}\DUrole{n}{kwargs}}}{}
A Bogoliubov quasiparticle.
\index{dispersion() (scdc.particle.Quasiparticle method)@\spxentry{dispersion()}\spxextra{scdc.particle.Quasiparticle method}}

\begin{fulllineitems}
\phantomsection\label{\detokenize{code_structure:scdc.particle.Quasiparticle.dispersion}}\pysiglinewithargsret{\sphinxbfcode{\sphinxupquote{dispersion}}}{\emph{\DUrole{n}{k}}}{}
Dispersion relation for quasiparticles.
\begin{quote}\begin{description}
\item[{Parameters}] \leavevmode
\sphinxstyleliteralstrong{\sphinxupquote{k}} (\sphinxstyleliteralemphasis{\sphinxupquote{float}}) \textendash{} wavenumber in material units.

\item[{Returns}] \leavevmode
quasiparticle energy in material units.

\item[{Return type}] \leavevmode
float

\end{description}\end{quote}

\end{fulllineitems}

\index{dispersion\_inverse() (scdc.particle.Quasiparticle method)@\spxentry{dispersion\_inverse()}\spxextra{scdc.particle.Quasiparticle method}}

\begin{fulllineitems}
\phantomsection\label{\detokenize{code_structure:scdc.particle.Quasiparticle.dispersion_inverse}}\pysiglinewithargsret{\sphinxbfcode{\sphinxupquote{dispersion\_inverse}}}{\emph{\DUrole{n}{E}}}{}
Inverse dispersion relation for quasiparticles.

Warning: the dispersion relation is not actually invertible. There are
two possible choices of a sign. If the sign \sphinxtitleref{s} is not specified, this
method selects it randomly!
\begin{quote}\begin{description}
\item[{Parameters}] \leavevmode\begin{itemize}
\item {} 
\sphinxstyleliteralstrong{\sphinxupquote{E}} (\sphinxstyleliteralemphasis{\sphinxupquote{float}}) \textendash{} quasiparticle energy in material units.

\item {} 
\sphinxstyleliteralstrong{\sphinxupquote{s}} (\sphinxstyleliteralemphasis{\sphinxupquote{int}}\sphinxstyleliteralemphasis{\sphinxupquote{, }}\sphinxstyleliteralemphasis{\sphinxupquote{optional}}) \textendash{} sign in the momentum solution.

\end{itemize}

\item[{Returns}] \leavevmode
quasiparticle wavenumber in material units.

\item[{Return type}] \leavevmode
float

\end{description}\end{quote}

\end{fulllineitems}

\index{ksign() (scdc.particle.Quasiparticle property)@\spxentry{ksign()}\spxextra{scdc.particle.Quasiparticle property}}

\begin{fulllineitems}
\phantomsection\label{\detokenize{code_structure:scdc.particle.Quasiparticle.ksign}}\pysigline{\sphinxbfcode{\sphinxupquote{property }}\sphinxbfcode{\sphinxupquote{ksign}}}
the sign as determined from energy and momentum.
\begin{quote}\begin{description}
\item[{Type}] \leavevmode
int

\end{description}\end{quote}

\end{fulllineitems}


\end{fulllineitems}



\subsection{Interaction objects}
\label{\detokenize{code_structure:module-scdc.interaction}}\label{\detokenize{code_structure:interaction-objects}}\index{module@\spxentry{module}!scdc.interaction@\spxentry{scdc.interaction}}\index{scdc.interaction@\spxentry{scdc.interaction}!module@\spxentry{module}}
This module defines the {\hyperref[\detokenize{code_structure:scdc.interaction.Interaction}]{\sphinxcrossref{\sphinxcode{\sphinxupquote{Interaction}}}}} class and subclasses.

This is where most of the action lives. Particles propagate from interaction
to interaction, and the interaction objects determine the final states that
arise at each point. The differential rates, for instance, are used here.
\index{DarkMatterScatter (class in scdc.interaction)@\spxentry{DarkMatterScatter}\spxextra{class in scdc.interaction}}

\begin{fulllineitems}
\phantomsection\label{\detokenize{code_structure:scdc.interaction.DarkMatterScatter}}\pysiglinewithargsret{\sphinxbfcode{\sphinxupquote{class }}\sphinxcode{\sphinxupquote{scdc.interaction.}}\sphinxbfcode{\sphinxupquote{DarkMatterScatter}}}{\emph{\DUrole{o}{*}\DUrole{n}{args}}, \emph{\DUrole{o}{**}\DUrole{n}{kwargs}}}{}
Represents an interaction in which DM scatters and creates two QPs.

This is a container, in the sense that no DM dynamics are actually
contained here. Rather, at initialization, two quasiparticles must be
supplied in the final state.
\begin{quote}\begin{description}
\item[{Parameters}] \leavevmode\begin{itemize}
\item {} 
\sphinxstyleliteralstrong{\sphinxupquote{qp1}} (\sphinxcode{\sphinxupquote{Quasiparticle}}) \textendash{} final state quasiparticle 1.

\item {} 
\sphinxstyleliteralstrong{\sphinxupquote{qp2}} (\sphinxcode{\sphinxupquote{Quasiparticle}}) \textendash{} final state quasiparticle 2.

\end{itemize}

\end{description}\end{quote}
\index{allowed() (scdc.interaction.DarkMatterScatter method)@\spxentry{allowed()}\spxextra{scdc.interaction.DarkMatterScatter method}}

\begin{fulllineitems}
\phantomsection\label{\detokenize{code_structure:scdc.interaction.DarkMatterScatter.allowed}}\pysiglinewithargsret{\sphinxbfcode{\sphinxupquote{allowed}}}{}{}
Determine whether the process is allowed for this initial state.
\begin{quote}\begin{description}
\item[{Returns}] \leavevmode
\sphinxtitleref{True} if the process can occur. Otherwise \sphinxtitleref{False}.

\item[{Return type}] \leavevmode
bool

\end{description}\end{quote}

\end{fulllineitems}


\end{fulllineitems}

\index{Interaction (class in scdc.interaction)@\spxentry{Interaction}\spxextra{class in scdc.interaction}}

\begin{fulllineitems}
\phantomsection\label{\detokenize{code_structure:scdc.interaction.Interaction}}\pysiglinewithargsret{\sphinxbfcode{\sphinxupquote{class }}\sphinxcode{\sphinxupquote{scdc.interaction.}}\sphinxbfcode{\sphinxupquote{Interaction}}}{\emph{\DUrole{n}{initial\_state}}, \emph{\DUrole{n}{material}}, \emph{\DUrole{o}{*}\DUrole{n}{args}}, \emph{\DUrole{o}{**}\DUrole{n}{kwargs}}}{}
Parent class for interactions.

An interaction combines an initial state, a material, and a final state.
Subclasses must provide the means to go from the initial state to the final
state.

Interactions are inherently treelike and can be treated as such. The
complication is that the initial and final states are \sphinxtitleref{Particle} objects,
not \sphinxtitleref{Interaction} objects. This necessitates the introduction of a new
container type, \sphinxtitleref{InteractionNode}, which specifies a particle and
the interaction in which it participates.
\begin{quote}\begin{description}
\item[{Parameters}] \leavevmode\begin{itemize}
\item {} 
\sphinxstyleliteralstrong{\sphinxupquote{initial\_state}} (\sphinxcode{\sphinxupquote{list}} of \sphinxcode{\sphinxupquote{Particle}}) \textendash{} particles in the
initial state. The initial state is always a list, even if it
contains only one object.

\item {} 
\sphinxstyleliteralstrong{\sphinxupquote{(}}\sphinxstyleliteralstrong{\sphinxupquote{obj}} (\sphinxstyleliteralemphasis{\sphinxupquote{material}}) \textendash{} \sphinxtitleref{Material}): the material in which this takes place.

\item {} 
\sphinxstyleliteralstrong{\sphinxupquote{n\_bins}} (\sphinxstyleliteralemphasis{\sphinxupquote{int}}\sphinxstyleliteralemphasis{\sphinxupquote{, }}\sphinxstyleliteralemphasis{\sphinxupquote{optional}}) \textendash{} number of bins to use for certain discretized
calculations. Defaults to 100.

\item {} 
\sphinxstyleliteralstrong{\sphinxupquote{final\_state}} (\sphinxcode{\sphinxupquote{list}} of \sphinxcode{\sphinxupquote{Particle}}, optional) \textendash{} particles in
the final state. If supplied, the interaction is treated as trivial
and the final state is never overwritten.

\item {} 
\sphinxstyleliteralstrong{\sphinxupquote{final}} (\sphinxstyleliteralemphasis{\sphinxupquote{bool}}\sphinxstyleliteralemphasis{\sphinxupquote{, }}\sphinxstyleliteralemphasis{\sphinxupquote{optional}}) \textendash{} if \sphinxtitleref{True}, the interaction is treated as
trivial and the final state is left unpopulated and unvalidated.
Defaults to \sphinxtitleref{False}.

\end{itemize}

\end{description}\end{quote}
\index{initial\_state (scdc.interaction.Interaction attribute)@\spxentry{initial\_state}\spxextra{scdc.interaction.Interaction attribute}}

\begin{fulllineitems}
\phantomsection\label{\detokenize{code_structure:scdc.interaction.Interaction.initial_state}}\pysigline{\sphinxbfcode{\sphinxupquote{initial\_state}}}
particles in the
initial state. The initial state is always a list, even if it
contains only one object.
\begin{quote}\begin{description}
\item[{Type}] \leavevmode
\sphinxcode{\sphinxupquote{list}} of \sphinxcode{\sphinxupquote{Particle}}

\end{description}\end{quote}

\end{fulllineitems}



\begin{fulllineitems}
\pysigline{\sphinxbfcode{\sphinxupquote{material~(obj}}}
\sphinxtitleref{Material}): the material in which this takes place.

\end{fulllineitems}

\index{final (scdc.interaction.Interaction attribute)@\spxentry{final}\spxextra{scdc.interaction.Interaction attribute}}

\begin{fulllineitems}
\phantomsection\label{\detokenize{code_structure:scdc.interaction.Interaction.final}}\pysigline{\sphinxbfcode{\sphinxupquote{final}}}
if \sphinxtitleref{True}, the interaction is treated as
trivial and the final state is left unpopulated and unvalidated.
\begin{quote}\begin{description}
\item[{Type}] \leavevmode
bool, optional

\end{description}\end{quote}

\end{fulllineitems}

\index{n\_initial (scdc.interaction.Interaction attribute)@\spxentry{n\_initial}\spxextra{scdc.interaction.Interaction attribute}}

\begin{fulllineitems}
\phantomsection\label{\detokenize{code_structure:scdc.interaction.Interaction.n_initial}}\pysigline{\sphinxbfcode{\sphinxupquote{n\_initial}}}
how many particles should be in the initial state.
\begin{quote}\begin{description}
\item[{Type}] \leavevmode
int

\end{description}\end{quote}

\end{fulllineitems}

\index{n\_final (scdc.interaction.Interaction attribute)@\spxentry{n\_final}\spxextra{scdc.interaction.Interaction attribute}}

\begin{fulllineitems}
\phantomsection\label{\detokenize{code_structure:scdc.interaction.Interaction.n_final}}\pysigline{\sphinxbfcode{\sphinxupquote{n\_final}}}
how many particles should be in the final state.
\begin{quote}\begin{description}
\item[{Type}] \leavevmode
int

\end{description}\end{quote}

\end{fulllineitems}

\index{initial (scdc.interaction.Interaction attribute)@\spxentry{initial}\spxextra{scdc.interaction.Interaction attribute}}

\begin{fulllineitems}
\phantomsection\label{\detokenize{code_structure:scdc.interaction.Interaction.initial}}\pysigline{\sphinxbfcode{\sphinxupquote{initial}}}
types of the initial particles,
listed in order of their appearance.
\begin{quote}\begin{description}
\item[{Type}] \leavevmode
\sphinxcode{\sphinxupquote{tuple}} of \sphinxtitleref{type}

\end{description}\end{quote}

\end{fulllineitems}

\index{final (scdc.interaction.Interaction attribute)@\spxentry{final}\spxextra{scdc.interaction.Interaction attribute}}

\begin{fulllineitems}
\phantomsection\label{\detokenize{code_structure:id5}}\pysigline{\sphinxbfcode{\sphinxupquote{final}}}
types of the final particles, listed in
order of their appearance.
\begin{quote}\begin{description}
\item[{Type}] \leavevmode
\sphinxcode{\sphinxupquote{tuple}} of \sphinxtitleref{type}

\end{description}\end{quote}

\end{fulllineitems}

\index{ip (scdc.interaction.Interaction attribute)@\spxentry{ip}\spxextra{scdc.interaction.Interaction attribute}}

\begin{fulllineitems}
\phantomsection\label{\detokenize{code_structure:scdc.interaction.Interaction.ip}}\pysigline{\sphinxbfcode{\sphinxupquote{ip}}}
first of the initial particles.
\begin{quote}\begin{description}
\item[{Type}] \leavevmode
\sphinxcode{\sphinxupquote{Particle}}

\end{description}\end{quote}

\end{fulllineitems}

\index{fixed\_final\_state (scdc.interaction.Interaction attribute)@\spxentry{fixed\_final\_state}\spxextra{scdc.interaction.Interaction attribute}}

\begin{fulllineitems}
\phantomsection\label{\detokenize{code_structure:scdc.interaction.Interaction.fixed_final_state}}\pysigline{\sphinxbfcode{\sphinxupquote{fixed\_final\_state}}}
\sphinxtitleref{True} if the final state was supplied at
initialization. In this case, the interaction is treated as trivial
and the final state is never overwritten.
\begin{quote}\begin{description}
\item[{Type}] \leavevmode
bool

\end{description}\end{quote}

\end{fulllineitems}

\index{allowed() (scdc.interaction.Interaction method)@\spxentry{allowed()}\spxextra{scdc.interaction.Interaction method}}

\begin{fulllineitems}
\phantomsection\label{\detokenize{code_structure:scdc.interaction.Interaction.allowed}}\pysiglinewithargsret{\sphinxbfcode{\sphinxupquote{allowed}}}{}{}
Determine whether the process is allowed for this initial state.
\begin{quote}\begin{description}
\item[{Returns}] \leavevmode
\sphinxtitleref{True} if the process can occur. Otherwise \sphinxtitleref{False}.

\item[{Return type}] \leavevmode
bool

\end{description}\end{quote}

\end{fulllineitems}

\index{interact() (scdc.interaction.Interaction method)@\spxentry{interact()}\spxextra{scdc.interaction.Interaction method}}

\begin{fulllineitems}
\phantomsection\label{\detokenize{code_structure:scdc.interaction.Interaction.interact}}\pysiglinewithargsret{\sphinxbfcode{\sphinxupquote{interact}}}{}{}
Interact, validate outcome, and set final state.

If the final state was supplied at initialization, do nothing.
\begin{quote}\begin{description}
\item[{Returns}] \leavevmode
the final state.

\item[{Return type}] \leavevmode
\sphinxcode{\sphinxupquote{list}} of \sphinxcode{\sphinxupquote{Particle}}

\end{description}\end{quote}

\end{fulllineitems}

\index{valid\_initial() (scdc.interaction.Interaction class method)@\spxentry{valid\_initial()}\spxextra{scdc.interaction.Interaction class method}}

\begin{fulllineitems}
\phantomsection\label{\detokenize{code_structure:scdc.interaction.Interaction.valid_initial}}\pysiglinewithargsret{\sphinxbfcode{\sphinxupquote{classmethod }}\sphinxbfcode{\sphinxupquote{valid\_initial}}}{\emph{\DUrole{n}{initial\_state}}}{}
Check that an initial state contains the correct particle types.
\begin{quote}\begin{description}
\item[{Parameters}] \leavevmode
\sphinxstyleliteralstrong{\sphinxupquote{initial\_state}} (\sphinxcode{\sphinxupquote{list}} of \sphinxcode{\sphinxupquote{Particle}}) \textendash{} candidate initial
state to validate.

\item[{Returns}] \leavevmode
\begin{description}
\item[{\sphinxtitleref{True} if the supplied initial state has the right particle}] \leavevmode
types for this interaction. Otherwise \sphinxtitleref{False}.

\end{description}


\item[{Return type}] \leavevmode
bool

\end{description}\end{quote}

\end{fulllineitems}


\end{fulllineitems}

\index{PhononDecayToQuasiparticles (class in scdc.interaction)@\spxentry{PhononDecayToQuasiparticles}\spxextra{class in scdc.interaction}}

\begin{fulllineitems}
\phantomsection\label{\detokenize{code_structure:scdc.interaction.PhononDecayToQuasiparticles}}\pysiglinewithargsret{\sphinxbfcode{\sphinxupquote{class }}\sphinxcode{\sphinxupquote{scdc.interaction.}}\sphinxbfcode{\sphinxupquote{PhononDecayToQuasiparticles}}}{\emph{\DUrole{o}{*}\DUrole{n}{args}}, \emph{\DUrole{o}{**}\DUrole{n}{kwargs}}}{}
The \sphinxtitleref{Interaction} for a phonon to decay to two quasiparticles.
\begin{quote}\begin{description}
\item[{Parameters}] \leavevmode
\sphinxstyleliteralstrong{\sphinxupquote{tolerance}} (\sphinxstyleliteralemphasis{\sphinxupquote{float}}\sphinxstyleliteralemphasis{\sphinxupquote{, }}\sphinxstyleliteralemphasis{\sphinxupquote{optional}}) \textendash{} a numerical tolerance for root\sphinxhyphen{}finding.
Defaults to 1e\sphinxhyphen{}3.

\end{description}\end{quote}
\index{tolerance (scdc.interaction.PhononDecayToQuasiparticles attribute)@\spxentry{tolerance}\spxextra{scdc.interaction.PhononDecayToQuasiparticles attribute}}

\begin{fulllineitems}
\phantomsection\label{\detokenize{code_structure:scdc.interaction.PhononDecayToQuasiparticles.tolerance}}\pysigline{\sphinxbfcode{\sphinxupquote{tolerance}}}
a numerical tolerance for root\sphinxhyphen{}finding.
\begin{quote}\begin{description}
\item[{Type}] \leavevmode
float, optional

\end{description}\end{quote}

\end{fulllineitems}

\index{allowed() (scdc.interaction.PhononDecayToQuasiparticles method)@\spxentry{allowed()}\spxextra{scdc.interaction.PhononDecayToQuasiparticles method}}

\begin{fulllineitems}
\phantomsection\label{\detokenize{code_structure:scdc.interaction.PhononDecayToQuasiparticles.allowed}}\pysiglinewithargsret{\sphinxbfcode{\sphinxupquote{allowed}}}{}{}
Determine whether the process is allowed.

Phonon decay can always take place if the energy of the phonon is at
least 2*Delta.
\begin{quote}\begin{description}
\item[{Returns}] \leavevmode
\sphinxtitleref{True} if phonon decay can occur. Otherwise \sphinxtitleref{False}.

\item[{Return type}] \leavevmode
bool

\end{description}\end{quote}

\end{fulllineitems}

\index{decay\_angles() (scdc.interaction.PhononDecayToQuasiparticles method)@\spxentry{decay\_angles()}\spxextra{scdc.interaction.PhononDecayToQuasiparticles method}}

\begin{fulllineitems}
\phantomsection\label{\detokenize{code_structure:scdc.interaction.PhononDecayToQuasiparticles.decay_angles}}\pysiglinewithargsret{\sphinxbfcode{\sphinxupquote{decay\_angles}}}{\emph{\DUrole{n}{Eqp1}}, \emph{\DUrole{n}{Eqp2}}, \emph{\DUrole{n}{s1}}, \emph{\DUrole{n}{s2}}}{}
Computes the relative angles of two QPs created in phonon decay.
\begin{quote}\begin{description}
\item[{Parameters}] \leavevmode\begin{itemize}
\item {} 
\sphinxstyleliteralstrong{\sphinxupquote{Eqp1}} (\sphinxstyleliteralemphasis{\sphinxupquote{float}}) \textendash{} energy of quasiparticle 1.

\item {} 
\sphinxstyleliteralstrong{\sphinxupquote{Eqp2}} (\sphinxstyleliteralemphasis{\sphinxupquote{float}}) \textendash{} energy of quasiparticle 2.

\item {} 
\sphinxstyleliteralstrong{\sphinxupquote{s1}} (\sphinxstyleliteralemphasis{\sphinxupquote{int}}) \textendash{} sign appearing in the inverse dispersion relation (QP1).

\item {} 
\sphinxstyleliteralstrong{\sphinxupquote{s2}} (\sphinxstyleliteralemphasis{\sphinxupquote{int}}) \textendash{} sign appearing in the inverse dispersion relation (QP2).

\end{itemize}

\item[{Returns}] \leavevmode
relative cos(theta) of quasiparticle 1.
float: relative cos(theta) of quasiparticle 2.

\item[{Return type}] \leavevmode
float

\end{description}\end{quote}

\end{fulllineitems}

\index{qp\_energy\_distribution() (scdc.interaction.PhononDecayToQuasiparticles method)@\spxentry{qp\_energy\_distribution()}\spxextra{scdc.interaction.PhononDecayToQuasiparticles method}}

\begin{fulllineitems}
\phantomsection\label{\detokenize{code_structure:scdc.interaction.PhononDecayToQuasiparticles.qp_energy_distribution}}\pysiglinewithargsret{\sphinxbfcode{\sphinxupquote{qp\_energy\_distribution}}}{\emph{\DUrole{n}{eps}\DUrole{o}{=}\DUrole{default_value}{1e\sphinxhyphen{}10}}}{}
Relative probability for a quasiparticle energy in phonon decay.

This is basically the integrand of eq. (27) of Kaplan+ 1976.
\begin{quote}\begin{description}
\item[{Parameters}] \leavevmode
\sphinxstyleliteralstrong{\sphinxupquote{eps}} (\sphinxstyleliteralemphasis{\sphinxupquote{float}}\sphinxstyleliteralemphasis{\sphinxupquote{, }}\sphinxstyleliteralemphasis{\sphinxupquote{optional}}) \textendash{} slight offset to prevent divide\sphinxhyphen{}by\sphinxhyphen{}zero
errors. Defaults to 1e\sphinxhyphen{}10.

\item[{Returns}] \leavevmode

QP energy bin centers.
\sphinxcode{\sphinxupquote{ndarray}} of \sphinxcode{\sphinxupquote{float}}: relative rate for each bin center,
\begin{quote}

normalized to unity.
\end{quote}


\item[{Return type}] \leavevmode
\sphinxcode{\sphinxupquote{ndarray}} of \sphinxcode{\sphinxupquote{float}}

\end{description}\end{quote}

\end{fulllineitems}


\end{fulllineitems}

\index{QuasiparticlePhononEmission (class in scdc.interaction)@\spxentry{QuasiparticlePhononEmission}\spxextra{class in scdc.interaction}}

\begin{fulllineitems}
\phantomsection\label{\detokenize{code_structure:scdc.interaction.QuasiparticlePhononEmission}}\pysiglinewithargsret{\sphinxbfcode{\sphinxupquote{class }}\sphinxcode{\sphinxupquote{scdc.interaction.}}\sphinxbfcode{\sphinxupquote{QuasiparticlePhononEmission}}}{\emph{\DUrole{o}{*}\DUrole{n}{args}}, \emph{\DUrole{o}{**}\DUrole{n}{kwargs}}}{}
The \sphinxtitleref{Interaction} for a quasiparticle to emit a phonon.
\begin{quote}\begin{description}
\item[{Parameters}] \leavevmode
\sphinxstyleliteralstrong{\sphinxupquote{tolerance}} (\sphinxstyleliteralemphasis{\sphinxupquote{float}}\sphinxstyleliteralemphasis{\sphinxupquote{, }}\sphinxstyleliteralemphasis{\sphinxupquote{optional}}) \textendash{} a numerical tolerance for root\sphinxhyphen{}finding.
Defaults to 1e\sphinxhyphen{}3.

\end{description}\end{quote}
\index{tolerance (scdc.interaction.QuasiparticlePhononEmission attribute)@\spxentry{tolerance}\spxextra{scdc.interaction.QuasiparticlePhononEmission attribute}}

\begin{fulllineitems}
\phantomsection\label{\detokenize{code_structure:scdc.interaction.QuasiparticlePhononEmission.tolerance}}\pysigline{\sphinxbfcode{\sphinxupquote{tolerance}}}
a numerical tolerance for root\sphinxhyphen{}finding.
\begin{quote}\begin{description}
\item[{Type}] \leavevmode
float, optional

\end{description}\end{quote}

\end{fulllineitems}

\index{allowed() (scdc.interaction.QuasiparticlePhononEmission method)@\spxentry{allowed()}\spxextra{scdc.interaction.QuasiparticlePhononEmission method}}

\begin{fulllineitems}
\phantomsection\label{\detokenize{code_structure:scdc.interaction.QuasiparticlePhononEmission.allowed}}\pysiglinewithargsret{\sphinxbfcode{\sphinxupquote{allowed}}}{\emph{\DUrole{n}{exact}\DUrole{o}{=}\DUrole{default_value}{False}}}{}
Determine whether the process is allowed for this initial state.
\begin{quote}\begin{description}
\item[{Returns}] \leavevmode
\sphinxtitleref{True} if the process can occur. Otherwise \sphinxtitleref{False}.

\item[{Return type}] \leavevmode
bool

\end{description}\end{quote}

\end{fulllineitems}

\index{final\_state\_angles() (scdc.interaction.QuasiparticlePhononEmission method)@\spxentry{final\_state\_angles()}\spxextra{scdc.interaction.QuasiparticlePhononEmission method}}

\begin{fulllineitems}
\phantomsection\label{\detokenize{code_structure:scdc.interaction.QuasiparticlePhononEmission.final_state_angles}}\pysiglinewithargsret{\sphinxbfcode{\sphinxupquote{final\_state\_angles}}}{\emph{\DUrole{n}{phonon\_energy}}, \emph{\DUrole{n}{sp}}}{}
Cosine of angle of final state particles after QP scattering.

Both angles are computed relative to the direction of the proximal
quasiparticle just before scattering, not necessarily the initial
quasiparticle.

DEBUGGING NOTE: this follows section “Better” of \sphinxtitleref{dispersion.nb}.
\begin{quote}\begin{description}
\item[{Parameters}] \leavevmode\begin{itemize}
\item {} 
\sphinxstyleliteralstrong{\sphinxupquote{phonon\_energy}} (\sphinxstyleliteralemphasis{\sphinxupquote{float}}) \textendash{} phonon energy in units of Delta.

\item {} 
\sphinxstyleliteralstrong{\sphinxupquote{qp\_energy}} (\sphinxstyleliteralemphasis{\sphinxupquote{float}}) \textendash{} Initial quasiparticle energy in units of Delta.

\item {} 
\sphinxstyleliteralstrong{\sphinxupquote{sp}} (\sphinxcode{\sphinxupquote{list}} of int) \textendash{} momentum sign for final quasiparticle.

\end{itemize}

\item[{Returns}] \leavevmode
cosine of the angle of the emitted quasiparticle.
float: cosine of the angle of the emitted phonon.

\item[{Return type}] \leavevmode
float

\end{description}\end{quote}

\end{fulllineitems}

\index{min\_cos\_deflection() (scdc.interaction.QuasiparticlePhononEmission method)@\spxentry{min\_cos\_deflection()}\spxextra{scdc.interaction.QuasiparticlePhononEmission method}}

\begin{fulllineitems}
\phantomsection\label{\detokenize{code_structure:scdc.interaction.QuasiparticlePhononEmission.min_cos_deflection}}\pysiglinewithargsret{\sphinxbfcode{\sphinxupquote{min\_cos\_deflection}}}{}{}
Max angular deflection, expressed as minimum of cos(theta).

For a given QP energy, there is a maximum amount of momentum it can
shed via phonon emission. This produces an upper bound on the angular
deflection of the QP. We express this in the form of a lower bound on
the cosine of the deflection angle.
\begin{quote}\begin{description}
\item[{Returns}] \leavevmode
min of cos(theta).

\item[{Return type}] \leavevmode
float

\end{description}\end{quote}

\end{fulllineitems}

\index{phonon\_energy\_distribution() (scdc.interaction.QuasiparticlePhononEmission method)@\spxentry{phonon\_energy\_distribution()}\spxextra{scdc.interaction.QuasiparticlePhononEmission method}}

\begin{fulllineitems}
\phantomsection\label{\detokenize{code_structure:scdc.interaction.QuasiparticlePhononEmission.phonon_energy_distribution}}\pysiglinewithargsret{\sphinxbfcode{\sphinxupquote{phonon\_energy\_distribution}}}{}{}
Relative phonon emission probability by a quasiparticle.

Computation follows dGamma/dx in eq. (21) of Noah’s note. We normalize
directly. If the distribution cannot be normalized (all values vanish),
both return values are \sphinxtitleref{None}.
\begin{quote}\begin{description}
\item[{Returns}] \leavevmode

phonon energy bin centers.
\sphinxcode{\sphinxupquote{ndarray}} of \sphinxcode{\sphinxupquote{float}}: relative rate for each bin center,
\begin{quote}

normalized to unity.
\end{quote}


\item[{Return type}] \leavevmode
\sphinxcode{\sphinxupquote{ndarray}} of \sphinxcode{\sphinxupquote{float}}

\item[{Raises}] \leavevmode
\sphinxstyleliteralstrong{\sphinxupquote{NonPhysicalException}} \textendash{} If the max phonon energy is not positive.

\end{description}\end{quote}

\end{fulllineitems}

\index{phonon\_energy\_region() (scdc.interaction.QuasiparticlePhononEmission method)@\spxentry{phonon\_energy\_region()}\spxextra{scdc.interaction.QuasiparticlePhononEmission method}}

\begin{fulllineitems}
\phantomsection\label{\detokenize{code_structure:scdc.interaction.QuasiparticlePhononEmission.phonon_energy_region}}\pysiglinewithargsret{\sphinxbfcode{\sphinxupquote{phonon\_energy\_region}}}{}{}
Min / max allowed phonon energy for emission by a quasiparticle.

The min and max energies are determined by finding the energy range
in which the final\sphinxhyphen{}state angles are physical, i.e.
\begin{quote}

\(|\cos(\theta)| < 1\).
\end{quote}

This needs to be done for both the quasiparticle and phonon angle.
Moreover, since there is a sign ambiguity in the quasiparticle
momentum, this needs to be done for each sign independently to get
self\sphinxhyphen{}consistent regions that work for both the quasiparticle angle and
the phonon angle.

Thus, we take the following approach. For each sign in turn, we find
the physical region for each of the two angles. The intersection of
these two regions is what we want. But then we have to do this for each
sign, and take the union of the resulting intersections.

In principle, this may be disjoint. For the moment, rather than
significantly change the architecture of the code, I will simply throw
an error if the regions are disjoint.
\begin{quote}\begin{description}
\item[{Parameters}] \leavevmode
\sphinxstyleliteralstrong{\sphinxupquote{qp\_energy}} (\sphinxstyleliteralemphasis{\sphinxupquote{float}}) \textendash{} Initial quasiparticle energy in units of Delta.

\item[{Returns}] \leavevmode
minimum emitted phonon energy in units of Delta.
float: maximum emitted phonon energy in units of Delta.

\item[{Return type}] \leavevmode
float

\end{description}\end{quote}

\end{fulllineitems}


\end{fulllineitems}



\subsection{Material objects}
\label{\detokenize{code_structure:module-scdc.material}}\label{\detokenize{code_structure:material-objects}}\index{module@\spxentry{module}!scdc.material@\spxentry{scdc.material}}\index{scdc.material@\spxentry{scdc.material}!module@\spxentry{module}}
This module defines a \sphinxtitleref{Material} class.

A note on units: at various places in the code, it is useful to have work with
non\sphinxhyphen{}dimensionalized quantities. However, picking these quantities requires
some care, because there are two sets of scales: those associated with
behavior near and far from the gap. We are mainly interested in near\sphinxhyphen{}gap
behavior, so we use the half\sphinxhyphen{}gap Delta as the energy scale. Now, excitations
with energy Delta have momentum k\_F, so it makes sense to take k\_F as our
preferred momentum scale. So far we have
\begin{quote}

Delta = k\_F = 1.
\end{quote}

Having made this choice, we automatically obtain a velocity v\_m and a mass m\_m
as follows:
\begin{quote}

v\_m = Delta/k\_F = 1,
m\_m = k\_F/v\_m = k\_F\textasciicircum{}2/Delta = 1.
\end{quote}

These parameters do NOT correspond to obvious scales in the material. In
particular, v\_m is distinct from the sound speed c\_s, and m\_m is distinct from
the carrier mass m\_star. In other words, c\_s\_m != 1 and m\_star\_m != 1.

We still need to be able to express a unit of length in order to put hbar into
these units. So far, however, time and length always appear in ratio. It
seems, then, that the natural thing to do is to simply set
\begin{quote}

hbar = 1.
\end{quote}

The natural units of time and length are then
\begin{quote}

t\_m = hbar/Delta = 1,
l\_m = hbar/Delta*v\_m = hbar*k\_F/Delta\textasciicircum{}2 = 1.
\end{quote}

It is useful to distinguish quantities in ordinary units from those in
material units. To that end, we will append “\_m” to the names of quantities
that are defined in material units.

For uniformity, ALL quantities not in material units are assumed to be in
units of an appropriate power of eV with c=hbar=kB=1.
\index{Material (class in scdc.material)@\spxentry{Material}\spxextra{class in scdc.material}}

\begin{fulllineitems}
\phantomsection\label{\detokenize{code_structure:scdc.material.Material}}\pysiglinewithargsret{\sphinxbfcode{\sphinxupquote{class }}\sphinxcode{\sphinxupquote{scdc.material.}}\sphinxbfcode{\sphinxupquote{Material}}}{\emph{\DUrole{o}{**}\DUrole{n}{kwargs}}}{}
Container class for material properties.
\begin{quote}\begin{description}
\item[{Parameters}] \leavevmode\begin{itemize}
\item {} 
\sphinxstyleliteralstrong{\sphinxupquote{symbol}} (\sphinxstyleliteralemphasis{\sphinxupquote{str}}) \textendash{} chemical symbol.

\item {} 
\sphinxstyleliteralstrong{\sphinxupquote{gamma}} (\sphinxstyleliteralemphasis{\sphinxupquote{float}}) \textendash{} gamma parameter.

\item {} 
\sphinxstyleliteralstrong{\sphinxupquote{c\_s}} (\sphinxstyleliteralemphasis{\sphinxupquote{float}}) \textendash{} speed of sound in m/s.

\item {} 
\sphinxstyleliteralstrong{\sphinxupquote{T\_c}} (\sphinxstyleliteralemphasis{\sphinxupquote{float}}) \textendash{} critical temperature in K.

\item {} 
\sphinxstyleliteralstrong{\sphinxupquote{Delta}} (\sphinxstyleliteralemphasis{\sphinxupquote{float}}) \textendash{} half of the gap in eV.

\item {} 
\sphinxstyleliteralstrong{\sphinxupquote{E\_F}} (\sphinxstyleliteralemphasis{\sphinxupquote{float}}) \textendash{} Fermi energy in eV.

\item {} 
\sphinxstyleliteralstrong{\sphinxupquote{m\_star}} (\sphinxstyleliteralemphasis{\sphinxupquote{float}}) \textendash{} effective mass? in units of m\_e.

\item {} 
\sphinxstyleliteralstrong{\sphinxupquote{beta}} (\sphinxstyleliteralemphasis{\sphinxupquote{float}}) \textendash{} beta.

\end{itemize}

\end{description}\end{quote}
\index{symbol (scdc.material.Material attribute)@\spxentry{symbol}\spxextra{scdc.material.Material attribute}}

\begin{fulllineitems}
\phantomsection\label{\detokenize{code_structure:scdc.material.Material.symbol}}\pysigline{\sphinxbfcode{\sphinxupquote{symbol}}}
chemical symbol.
\begin{quote}\begin{description}
\item[{Type}] \leavevmode
str

\end{description}\end{quote}

\end{fulllineitems}

\index{gamma (scdc.material.Material attribute)@\spxentry{gamma}\spxextra{scdc.material.Material attribute}}

\begin{fulllineitems}
\phantomsection\label{\detokenize{code_structure:scdc.material.Material.gamma}}\pysigline{\sphinxbfcode{\sphinxupquote{gamma}}}
gamma parameter.
\begin{quote}\begin{description}
\item[{Type}] \leavevmode
float

\end{description}\end{quote}

\end{fulllineitems}

\index{c\_s (scdc.material.Material attribute)@\spxentry{c\_s}\spxextra{scdc.material.Material attribute}}

\begin{fulllineitems}
\phantomsection\label{\detokenize{code_structure:scdc.material.Material.c_s}}\pysigline{\sphinxbfcode{\sphinxupquote{c\_s}}}
speed of sound in m/s.
\begin{quote}\begin{description}
\item[{Type}] \leavevmode
float

\end{description}\end{quote}

\end{fulllineitems}

\index{T\_c (scdc.material.Material attribute)@\spxentry{T\_c}\spxextra{scdc.material.Material attribute}}

\begin{fulllineitems}
\phantomsection\label{\detokenize{code_structure:scdc.material.Material.T_c}}\pysigline{\sphinxbfcode{\sphinxupquote{T\_c}}}
critical temperature in K.
\begin{quote}\begin{description}
\item[{Type}] \leavevmode
float

\end{description}\end{quote}

\end{fulllineitems}

\index{Delta (scdc.material.Material attribute)@\spxentry{Delta}\spxextra{scdc.material.Material attribute}}

\begin{fulllineitems}
\phantomsection\label{\detokenize{code_structure:scdc.material.Material.Delta}}\pysigline{\sphinxbfcode{\sphinxupquote{Delta}}}
half of the gap.
\begin{quote}\begin{description}
\item[{Type}] \leavevmode
float

\end{description}\end{quote}

\end{fulllineitems}

\index{E\_F (scdc.material.Material attribute)@\spxentry{E\_F}\spxextra{scdc.material.Material attribute}}

\begin{fulllineitems}
\phantomsection\label{\detokenize{code_structure:scdc.material.Material.E_F}}\pysigline{\sphinxbfcode{\sphinxupquote{E\_F}}}
Fermi energy in eV.
\begin{quote}\begin{description}
\item[{Type}] \leavevmode
float

\end{description}\end{quote}

\end{fulllineitems}

\index{m\_star (scdc.material.Material attribute)@\spxentry{m\_star}\spxextra{scdc.material.Material attribute}}

\begin{fulllineitems}
\phantomsection\label{\detokenize{code_structure:scdc.material.Material.m_star}}\pysigline{\sphinxbfcode{\sphinxupquote{m\_star}}}
effective mass? in units of m\_e.
\begin{quote}\begin{description}
\item[{Type}] \leavevmode
float

\end{description}\end{quote}

\end{fulllineitems}

\index{z (scdc.material.Material attribute)@\spxentry{z}\spxextra{scdc.material.Material attribute}}

\begin{fulllineitems}
\phantomsection\label{\detokenize{code_structure:scdc.material.Material.z}}\pysigline{\sphinxbfcode{\sphinxupquote{z}}}
Fermi energy in units of Delta.
\begin{quote}\begin{description}
\item[{Type}] \leavevmode
float

\end{description}\end{quote}

\end{fulllineitems}

\index{beta (scdc.material.Material attribute)@\spxentry{beta}\spxextra{scdc.material.Material attribute}}

\begin{fulllineitems}
\phantomsection\label{\detokenize{code_structure:scdc.material.Material.beta}}\pysigline{\sphinxbfcode{\sphinxupquote{beta}}}
beta.
\begin{quote}\begin{description}
\item[{Type}] \leavevmode
float

\end{description}\end{quote}

\end{fulllineitems}

\index{mcs2 (scdc.material.Material attribute)@\spxentry{mcs2}\spxextra{scdc.material.Material attribute}}

\begin{fulllineitems}
\phantomsection\label{\detokenize{code_structure:scdc.material.Material.mcs2}}\pysigline{\sphinxbfcode{\sphinxupquote{mcs2}}}
m*c\_s\textasciicircum{}2 in units of eV.
\begin{quote}\begin{description}
\item[{Type}] \leavevmode
float

\end{description}\end{quote}

\end{fulllineitems}

\index{coherence\_uvvu() (scdc.material.Material method)@\spxentry{coherence\_uvvu()}\spxextra{scdc.material.Material method}}

\begin{fulllineitems}
\phantomsection\label{\detokenize{code_structure:scdc.material.Material.coherence_uvvu}}\pysiglinewithargsret{\sphinxbfcode{\sphinxupquote{coherence\_uvvu}}}{\emph{\DUrole{n}{s}}, \emph{\DUrole{n}{k1}}, \emph{\DUrole{n}{k2}}}{}
The coherence factor (uv’ +/\sphinxhyphen{} vu’)\textasciicircum{}2. The sign is \sphinxtitleref{s}.

Note that \sphinxtitleref{k1} and \sphinxtitleref{k2} are assumed to be in material units (i.e. in
units of k\_F).
\begin{quote}\begin{description}
\item[{Parameters}] \leavevmode\begin{itemize}
\item {} 
\sphinxstyleliteralstrong{\sphinxupquote{s}} (\sphinxstyleliteralemphasis{\sphinxupquote{int}}) \textendash{} sign that appears in the coherence factor (1 or \sphinxhyphen{}1).

\item {} 
\sphinxstyleliteralstrong{\sphinxupquote{k1}} (\sphinxstyleliteralemphasis{\sphinxupquote{float}}) \textendash{} momentum of quasiparticle 1 in material units.

\item {} 
\sphinxstyleliteralstrong{\sphinxupquote{k2}} (\sphinxstyleliteralemphasis{\sphinxupquote{float}}) \textendash{} momentum of quasiparticle 2 in material units.

\end{itemize}

\item[{Returns}] \leavevmode
the coherence factor (uv’ +/\sphinxhyphen{} vu’)\textasciicircum{}2.

\item[{Return type}] \leavevmode
float

\end{description}\end{quote}

\end{fulllineitems}

\index{epsilon\_lindhard() (scdc.material.Material method)@\spxentry{epsilon\_lindhard()}\spxextra{scdc.material.Material method}}

\begin{fulllineitems}
\phantomsection\label{\detokenize{code_structure:scdc.material.Material.epsilon_lindhard}}\pysiglinewithargsret{\sphinxbfcode{\sphinxupquote{epsilon\_lindhard}}}{\emph{\DUrole{n}{q}}, \emph{\DUrole{n}{omega}}}{}
Lindhard dielectric function.

\end{fulllineitems}

\index{info() (scdc.material.Material method)@\spxentry{info()}\spxextra{scdc.material.Material method}}

\begin{fulllineitems}
\phantomsection\label{\detokenize{code_structure:scdc.material.Material.info}}\pysiglinewithargsret{\sphinxbfcode{\sphinxupquote{info}}}{}{}
Material data in a json\sphinxhyphen{}serializable format.
\begin{quote}\begin{description}
\item[{Returns}] \leavevmode
material info.

\item[{Return type}] \leavevmode
dict

\end{description}\end{quote}

\end{fulllineitems}

\index{qpe() (scdc.material.Material method)@\spxentry{qpe()}\spxextra{scdc.material.Material method}}

\begin{fulllineitems}
\phantomsection\label{\detokenize{code_structure:scdc.material.Material.qpe}}\pysiglinewithargsret{\sphinxbfcode{\sphinxupquote{qpe}}}{\emph{\DUrole{n}{k}}}{}
Compute the energy of a quasiparticle of momentum \(k\).

This is a convenience method that is identical to the \sphinxcode{\sphinxupquote{Quasiparticle}}
dispersion relation. In fact, the two might be merged later on…
\begin{quote}\begin{description}
\item[{Parameters}] \leavevmode
\sphinxstyleliteralstrong{\sphinxupquote{k}} \textendash{} QP momentum

\item[{Returns}] \leavevmode
QP energy in material units

\item[{Return type}] \leavevmode
float

\end{description}\end{quote}

\end{fulllineitems}


\end{fulllineitems}



\subsection{Initial distribution}
\label{\detokenize{code_structure:module-scdc.initial.distribution.integral}}\label{\detokenize{code_structure:initial-distribution}}\index{module@\spxentry{module}!scdc.initial.distribution.integral@\spxentry{scdc.initial.distribution.integral}}\index{scdc.initial.distribution.integral@\spxentry{scdc.initial.distribution.integral}!module@\spxentry{module}}
This module defines the distribution of the initial excitations in angles
and momenta, as a function of the scattering matrix element. Unlike prior
implementations using MCMC methods, here the sampling is carried by direct
integration of the distribution.
\index{InitialSampler (class in scdc.initial.distribution.integral)@\spxentry{InitialSampler}\spxextra{class in scdc.initial.distribution.integral}}

\begin{fulllineitems}
\phantomsection\label{\detokenize{code_structure:scdc.initial.distribution.integral.InitialSampler}}\pysiglinewithargsret{\sphinxbfcode{\sphinxupquote{class }}\sphinxcode{\sphinxupquote{scdc.initial.distribution.integral.}}\sphinxbfcode{\sphinxupquote{InitialSampler}}}{\emph{\DUrole{o}{*}\DUrole{n}{args}}, \emph{\DUrole{o}{**}\DUrole{n}{kwargs}}}{}
Sampler for parameter distribution of initial excitations.
\begin{description}
\item[{Here’s the general strategy:}] \leavevmode\begin{enumerate}
\sphinxsetlistlabels{\arabic}{enumi}{enumii}{}{.}%
\item {} \begin{description}
\item[{Start by computing the total rate for each of many speeds.}] \leavevmode\begin{itemize}
\item {} 
Do this by computing the total rate on a sparse grid of speeds
and interpolating.

\item {} 
The total rate calculation at each speed requires that we
calculate the rate at many values in the (cq, rq) plane. Store
these values for later.

\end{itemize}

\end{description}

\item {} 
Use these together with the marginal distribution of speeds in the
halo to sample a set of DM speeds.

\item {} 
For each DM speed, interpolate the rates between different slices of
the (cq, rq) plane, corresponding to different discrete speeds.
\begin{itemize}
\item {} 
This gives an interpolated set of rates in the (cq, rq) plane at
the speed of interest.

\end{itemize}

\item {} 
Interpolate these rates to sample (cq, rq).

\item {} \begin{description}
\item[{Determine two remaining kinematical variables.}] \leavevmode\begin{itemize}
\item {} 
The azimuthal angle of the final states with respect to the
momentum transfer can be sampled uniformly.

\item {} 
The angle of the incoming DM with respect to the DM wind axis
has a well\sphinxhyphen{}determined marginal distribution at fixed DM speed.

\end{itemize}

\end{description}

\end{enumerate}

\end{description}

Args:

Attributes:
\begin{quote}\begin{description}
\item[{Raises}] \leavevmode
\sphinxstyleliteralstrong{\sphinxupquote{NotAllowedException}} \textendash{} DM mass is below threshold for scattering for all
    available velocities.

\end{description}\end{quote}
\index{ensemble() (scdc.initial.distribution.integral.InitialSampler method)@\spxentry{ensemble()}\spxextra{scdc.initial.distribution.integral.InitialSampler method}}

\begin{fulllineitems}
\phantomsection\label{\detokenize{code_structure:scdc.initial.distribution.integral.InitialSampler.ensemble}}\pysiglinewithargsret{\sphinxbfcode{\sphinxupquote{ensemble}}}{\emph{\DUrole{o}{*}\DUrole{n}{args}}, \emph{\DUrole{o}{**}\DUrole{n}{kwargs}}}{}
Sample an ensemble of DM scatters.
\begin{quote}\begin{description}
\item[{Parameters}] \leavevmode
\sphinxstyleliteralstrong{\sphinxupquote{n\_samples}} (\sphinxstyleliteralemphasis{\sphinxupquote{int}}) \textendash{} number of samples to draw.

\item[{Returns}] \leavevmode
an ensemble of \sphinxcode{\sphinxupquote{DarkMatterScatter}} events.

\item[{Return type}] \leavevmode
\sphinxcode{\sphinxupquote{Ensemble}}

\end{description}\end{quote}

\end{fulllineitems}

\index{q\_rate\_grid() (scdc.initial.distribution.integral.InitialSampler method)@\spxentry{q\_rate\_grid()}\spxextra{scdc.initial.distribution.integral.InitialSampler method}}

\begin{fulllineitems}
\phantomsection\label{\detokenize{code_structure:scdc.initial.distribution.integral.InitialSampler.q_rate_grid}}\pysiglinewithargsret{\sphinxbfcode{\sphinxupquote{q\_rate\_grid}}}{\emph{\DUrole{n}{r1}}, \emph{\DUrole{n}{order}\DUrole{o}{=}\DUrole{default_value}{20}}}{}
Compute a grid of rates at fixed \((c_q,r_q)\).
\begin{quote}\begin{description}
\item[{Parameters}] \leavevmode\begin{itemize}
\item {} 
\sphinxstyleliteralstrong{\sphinxupquote{r1}} (\sphinxstyleliteralemphasis{\sphinxupquote{float}}) \textendash{} fixed \(r_1\) value.

\item {} 
\sphinxstyleliteralstrong{\sphinxupquote{order}} (\sphinxstyleliteralemphasis{\sphinxupquote{int}}\sphinxstyleliteralemphasis{\sphinxupquote{, }}\sphinxstyleliteralemphasis{\sphinxupquote{optional}}) \textendash{} fixed order to use for Gaussian quadrature.

\item {} 
\sphinxstyleliteralstrong{\sphinxupquote{threshold}} (\sphinxstyleliteralemphasis{\sphinxupquote{float}}\sphinxstyleliteralemphasis{\sphinxupquote{, }}\sphinxstyleliteralemphasis{\sphinxupquote{optional}}) \textendash{} the fraction of the integrated rate
that should be preserved when doing cuts. Note that cuts are
repeated \sphinxcode{\sphinxupquote{n\_cuts}} times. Defaults to 0.99.

\end{itemize}

\item[{Returns}] \leavevmode
1d array of unique \(c_q\) values.
ndarray: 1d array of unique \(r_q\) values.
ndarray: 2d array of rate values.

\item[{Return type}] \leavevmode
ndarray

\item[{Raises}] \leavevmode\begin{itemize}
\item {} 
\sphinxstyleliteralstrong{\sphinxupquote{ValueError}} \textendash{} if all rates are zero.

\item {} 
\sphinxstyleliteralstrong{\sphinxupquote{ValueError}} \textendash{} if \(r_1\) is too small for any scattering to be
    kinematically allowed.

\end{itemize}

\end{description}\end{quote}

\end{fulllineitems}

\index{q\_to\_p() (scdc.initial.distribution.integral.InitialSampler method)@\spxentry{q\_to\_p()}\spxextra{scdc.initial.distribution.integral.InitialSampler method}}

\begin{fulllineitems}
\phantomsection\label{\detokenize{code_structure:scdc.initial.distribution.integral.InitialSampler.q_to_p}}\pysiglinewithargsret{\sphinxbfcode{\sphinxupquote{q\_to\_p}}}{\emph{\DUrole{n}{r1}}, \emph{\DUrole{n}{rq}}, \emph{\DUrole{n}{r3}}, \emph{\DUrole{n}{c1}}, \emph{\DUrole{n}{cq}}}{}
Transform momentum\sphinxhyphen{}transfer samples to get the final\sphinxhyphen{}state momenta.

Note that this method also samples azimuthal angles for the
quasiparticles with respect to the momentum transfer. The three input
arrays must have the same length.
\begin{quote}\begin{description}
\item[{Parameters}] \leavevmode\begin{itemize}
\item {} 
\sphinxstyleliteralstrong{\sphinxupquote{r1}} (\sphinxstyleliteralemphasis{\sphinxupquote{ndarray}}) \textendash{} \(r_1\) values.

\item {} 
\sphinxstyleliteralstrong{\sphinxupquote{rq}} (\sphinxstyleliteralemphasis{\sphinxupquote{ndarray}}) \textendash{} \(r_q\) values.

\item {} 
\sphinxstyleliteralstrong{\sphinxupquote{r3}} (\sphinxstyleliteralemphasis{\sphinxupquote{ndarray}}) \textendash{} \(r_3\) values.

\item {} 
\sphinxstyleliteralstrong{\sphinxupquote{c1}} (\sphinxstyleliteralemphasis{\sphinxupquote{ndarray}}) \textendash{} \(c_1\) values.

\item {} 
\sphinxstyleliteralstrong{\sphinxupquote{cq}} (\sphinxstyleliteralemphasis{\sphinxupquote{ndarray}}) \textendash{} \(\cos\theta_q\) values.

\end{itemize}

\item[{Returns}] \leavevmode
\begin{description}
\item[{an array of samples of shape \sphinxcode{\sphinxupquote{(len(cq), 4)}}. These}] \leavevmode
samples have the form \((r_3, r_4, c_3, c_4)\).

\end{description}


\item[{Return type}] \leavevmode
ndarray

\end{description}\end{quote}

\end{fulllineitems}

\index{sample() (scdc.initial.distribution.integral.InitialSampler method)@\spxentry{sample()}\spxextra{scdc.initial.distribution.integral.InitialSampler method}}

\begin{fulllineitems}
\phantomsection\label{\detokenize{code_structure:scdc.initial.distribution.integral.InitialSampler.sample}}\pysiglinewithargsret{\sphinxbfcode{\sphinxupquote{sample}}}{\emph{\DUrole{n}{n\_samples}}, \emph{\DUrole{n}{max\_attempts}\DUrole{o}{=}\DUrole{default_value}{None}}}{}
Keeps drawing samples until we get the desired number.

This method wraps \sphinxcode{\sphinxupquote{\_sample}} and has identical arguments and returns,
except for one additional argument that specifies the max number of
attempts.
\begin{quote}\begin{description}
\item[{Parameters}] \leavevmode\begin{itemize}
\item {} 
\sphinxstyleliteralstrong{\sphinxupquote{n\_samples}} (\sphinxstyleliteralemphasis{\sphinxupquote{int}}) \textendash{} number of samples to draw.

\item {} 
\sphinxstyleliteralstrong{\sphinxupquote{max\_attempts}} (\sphinxstyleliteralemphasis{\sphinxupquote{int}}) \textendash{} max number of calls to \sphinxcode{\sphinxupquote{\_sample}}. If
\sphinxcode{\sphinxupquote{None}}, never give up. Defaults to \sphinxcode{\sphinxupquote{None}}.

\end{itemize}

\item[{Returns}] \leavevmode
\(r_1\) values.
ndarray: \(r_2\) values.
ndarray: \(r_3\) values.
ndarray: \(r_4\) values.
ndarray: \(c_1\) values.
ndarray: \(c_2\) values.
ndarray: \(c_3\) values.
ndarray: \(c_4\) values.

\item[{Return type}] \leavevmode
ndarray

\end{description}\end{quote}

\end{fulllineitems}

\index{samples\_to\_ensemble() (scdc.initial.distribution.integral.InitialSampler method)@\spxentry{samples\_to\_ensemble()}\spxextra{scdc.initial.distribution.integral.InitialSampler method}}

\begin{fulllineitems}
\phantomsection\label{\detokenize{code_structure:scdc.initial.distribution.integral.InitialSampler.samples_to_ensemble}}\pysiglinewithargsret{\sphinxbfcode{\sphinxupquote{samples\_to\_ensemble}}}{\emph{\DUrole{n}{r1}}, \emph{\DUrole{n}{r2}}, \emph{\DUrole{n}{r3}}, \emph{\DUrole{n}{r4}}, \emph{\DUrole{n}{c1}}, \emph{\DUrole{n}{c2}}, \emph{\DUrole{n}{c3}}, \emph{\DUrole{n}{c4}}}{}
Convert sampled final\sphinxhyphen{}state momenta to an \sphinxcode{\sphinxupquote{Ensemble}}.

All input arrays must have the same shape.
\begin{quote}\begin{description}
\item[{Parameters}] \leavevmode\begin{itemize}
\item {} 
\sphinxstyleliteralstrong{\sphinxupquote{r1}} (\sphinxstyleliteralemphasis{\sphinxupquote{ndarray}}) \textendash{} sampled \(r_1\) values.

\item {} 
\sphinxstyleliteralstrong{\sphinxupquote{r1}} \textendash{} sampled \(r_2\) values.

\item {} 
\sphinxstyleliteralstrong{\sphinxupquote{r3}} (\sphinxstyleliteralemphasis{\sphinxupquote{ndarray}}) \textendash{} sampled \(r_3\) values.

\item {} 
\sphinxstyleliteralstrong{\sphinxupquote{r4}} (\sphinxstyleliteralemphasis{\sphinxupquote{ndarray}}) \textendash{} sampled \(r_4\) values.

\item {} 
\sphinxstyleliteralstrong{\sphinxupquote{c1}} (\sphinxstyleliteralemphasis{\sphinxupquote{ndarray}}) \textendash{} sampled \(c_1\) values.

\item {} 
\sphinxstyleliteralstrong{\sphinxupquote{c1}} \textendash{} sampled \(c_2\) values.

\item {} 
\sphinxstyleliteralstrong{\sphinxupquote{c3}} (\sphinxstyleliteralemphasis{\sphinxupquote{ndarray}}) \textendash{} sampled \(c_3\) values.

\item {} 
\sphinxstyleliteralstrong{\sphinxupquote{c4}} (\sphinxstyleliteralemphasis{\sphinxupquote{ndarray}}) \textendash{} sampled \(c_4\) values.

\end{itemize}

\item[{Returns}] \leavevmode
an ensemble of \sphinxcode{\sphinxupquote{DarkMatterScatter}} events.

\item[{Return type}] \leavevmode
\sphinxcode{\sphinxupquote{Ensemble}}

\end{description}\end{quote}

\end{fulllineitems}


\end{fulllineitems}

\index{RateIntegrator (class in scdc.initial.distribution.integral)@\spxentry{RateIntegrator}\spxextra{class in scdc.initial.distribution.integral}}

\begin{fulllineitems}
\phantomsection\label{\detokenize{code_structure:scdc.initial.distribution.integral.RateIntegrator}}\pysiglinewithargsret{\sphinxbfcode{\sphinxupquote{class }}\sphinxcode{\sphinxupquote{scdc.initial.distribution.integral.}}\sphinxbfcode{\sphinxupquote{RateIntegrator}}}{\emph{\DUrole{n}{m1}}, \emph{\DUrole{n}{matrix\_element}}, \emph{\DUrole{n}{material}}, \emph{\DUrole{n}{response}}, \emph{\DUrole{n}{vdf}}, \emph{\DUrole{o}{**}\DUrole{n}{kwargs}}}{}
Base class for integrating rates
\begin{quote}\begin{description}
\item[{Parameters}] \leavevmode\begin{itemize}
\item {} 
\sphinxstyleliteralstrong{\sphinxupquote{m1}} (\sphinxstyleliteralemphasis{\sphinxupquote{float}}) \textendash{} mass of the incoming particle.

\item {} 
\sphinxstyleliteralstrong{\sphinxupquote{matrix\_element}} (\sphinxstyleliteralemphasis{\sphinxupquote{callable}}) \textendash{} the matrix element for the scattering
process, as a function of the magnitude of the 3\sphinxhyphen{}momentum transfer.
A \sphinxcode{\sphinxupquote{ScatteringMatrixElement}} object can be supplied here.

\item {} 
\sphinxstyleliteralstrong{\sphinxupquote{material}} (\sphinxcode{\sphinxupquote{Material}}) \textendash{} a material object.

\item {} 
\sphinxstyleliteralstrong{\sphinxupquote{response}} (\sphinxstyleliteralemphasis{\sphinxupquote{callable}}) \textendash{} a function of \(q\) and \(\omega\)
characterizing material response.

\item {} 
\sphinxstyleliteralstrong{\sphinxupquote{vdf}} (\sphinxstyleliteralemphasis{\sphinxupquote{callable}}) \textendash{} a probability distribution for the velocity as a
function of zero, one, or two parameters. The number of arguments
must match the length of the argument \sphinxcode{\sphinxupquote{v\_args}} to \sphinxcode{\sphinxupquote{likelihood}}.

\end{itemize}

\end{description}\end{quote}
\index{m1 (scdc.initial.distribution.integral.RateIntegrator attribute)@\spxentry{m1}\spxextra{scdc.initial.distribution.integral.RateIntegrator attribute}}

\begin{fulllineitems}
\phantomsection\label{\detokenize{code_structure:scdc.initial.distribution.integral.RateIntegrator.m1}}\pysigline{\sphinxbfcode{\sphinxupquote{m1}}}
mass of the incoming particle.
\begin{quote}\begin{description}
\item[{Type}] \leavevmode
float

\end{description}\end{quote}

\end{fulllineitems}

\index{matrix\_element (scdc.initial.distribution.integral.RateIntegrator attribute)@\spxentry{matrix\_element}\spxextra{scdc.initial.distribution.integral.RateIntegrator attribute}}

\begin{fulllineitems}
\phantomsection\label{\detokenize{code_structure:scdc.initial.distribution.integral.RateIntegrator.matrix_element}}\pysigline{\sphinxbfcode{\sphinxupquote{matrix\_element}}}
the matrix element for the scattering
process, as a function of the magnitude of the 3\sphinxhyphen{}momentum transfer.
A \sphinxcode{\sphinxupquote{ScatteringMatrixElement}} object can be supplied here.
\begin{quote}\begin{description}
\item[{Type}] \leavevmode
callable

\end{description}\end{quote}

\end{fulllineitems}

\index{material (scdc.initial.distribution.integral.RateIntegrator attribute)@\spxentry{material}\spxextra{scdc.initial.distribution.integral.RateIntegrator attribute}}

\begin{fulllineitems}
\phantomsection\label{\detokenize{code_structure:scdc.initial.distribution.integral.RateIntegrator.material}}\pysigline{\sphinxbfcode{\sphinxupquote{material}}}
a material object.
\begin{quote}\begin{description}
\item[{Type}] \leavevmode
\sphinxcode{\sphinxupquote{Material}}

\end{description}\end{quote}

\end{fulllineitems}

\index{response (scdc.initial.distribution.integral.RateIntegrator attribute)@\spxentry{response}\spxextra{scdc.initial.distribution.integral.RateIntegrator attribute}}

\begin{fulllineitems}
\phantomsection\label{\detokenize{code_structure:scdc.initial.distribution.integral.RateIntegrator.response}}\pysigline{\sphinxbfcode{\sphinxupquote{response}}}
a function of \(q\) and \(\omega\)
characterizing material response.
\begin{quote}\begin{description}
\item[{Type}] \leavevmode
callable

\end{description}\end{quote}

\end{fulllineitems}

\index{vdf (scdc.initial.distribution.integral.RateIntegrator attribute)@\spxentry{vdf}\spxextra{scdc.initial.distribution.integral.RateIntegrator attribute}}

\begin{fulllineitems}
\phantomsection\label{\detokenize{code_structure:scdc.initial.distribution.integral.RateIntegrator.vdf}}\pysigline{\sphinxbfcode{\sphinxupquote{vdf}}}
a probability distribution for the velocity as a
function of zero, one, or two parameters. The number of arguments
must match the length of the argument \sphinxcode{\sphinxupquote{v\_args}} to \sphinxcode{\sphinxupquote{likelihood}}.
\begin{quote}\begin{description}
\item[{Type}] \leavevmode
callable

\end{description}\end{quote}

\end{fulllineitems}

\index{pdf() (scdc.initial.distribution.integral.RateIntegrator method)@\spxentry{pdf()}\spxextra{scdc.initial.distribution.integral.RateIntegrator method}}

\begin{fulllineitems}
\phantomsection\label{\detokenize{code_structure:scdc.initial.distribution.integral.RateIntegrator.pdf}}\pysiglinewithargsret{\sphinxbfcode{\sphinxupquote{pdf}}}{\emph{\DUrole{n}{r1}}, \emph{\DUrole{n}{rq}}, \emph{\DUrole{n}{cq}}, \emph{\DUrole{n}{r3}}}{}
Compute the differential rate summing over sign choices.
\begin{quote}\begin{description}
\item[{Parameters}] \leavevmode\begin{itemize}
\item {} 
\sphinxstyleliteralstrong{\sphinxupquote{r1}} (\sphinxstyleliteralemphasis{\sphinxupquote{float}}) \textendash{} \(r_1\).

\item {} 
\sphinxstyleliteralstrong{\sphinxupquote{rq}} (\sphinxstyleliteralemphasis{\sphinxupquote{float}}) \textendash{} \(r_q\).

\item {} 
\sphinxstyleliteralstrong{\sphinxupquote{cq}} (\sphinxstyleliteralemphasis{\sphinxupquote{float}}) \textendash{} \(c_q\).

\item {} 
\sphinxstyleliteralstrong{\sphinxupquote{r3}} (\sphinxstyleliteralemphasis{\sphinxupquote{float}}) \textendash{} \(r_3\).

\end{itemize}

\item[{Returns}] \leavevmode
differential rate (non\sphinxhyphen{}normalized).

\item[{Return type}] \leavevmode
float

\end{description}\end{quote}

\end{fulllineitems}

\index{pdf\_fixed\_sign() (scdc.initial.distribution.integral.RateIntegrator method)@\spxentry{pdf\_fixed\_sign()}\spxextra{scdc.initial.distribution.integral.RateIntegrator method}}

\begin{fulllineitems}
\phantomsection\label{\detokenize{code_structure:scdc.initial.distribution.integral.RateIntegrator.pdf_fixed_sign}}\pysiglinewithargsret{\sphinxbfcode{\sphinxupquote{pdf\_fixed\_sign}}}{\emph{\DUrole{n}{r1}}, \emph{\DUrole{n}{rq}}, \emph{\DUrole{n}{cq}}, \emph{\DUrole{n}{r3}}, \emph{\DUrole{n}{s}}}{}
Compute the pdf for fixed sign \(s\).
\begin{quote}\begin{description}
\item[{Parameters}] \leavevmode\begin{itemize}
\item {} 
\sphinxstyleliteralstrong{\sphinxupquote{r1}} (\sphinxstyleliteralemphasis{\sphinxupquote{float}}) \textendash{} \(r_1\).

\item {} 
\sphinxstyleliteralstrong{\sphinxupquote{rq}} (\sphinxstyleliteralemphasis{\sphinxupquote{float}}) \textendash{} \(r_q\).

\item {} 
\sphinxstyleliteralstrong{\sphinxupquote{cq}} (\sphinxstyleliteralemphasis{\sphinxupquote{float}}) \textendash{} \(c_q\).

\item {} 
\sphinxstyleliteralstrong{\sphinxupquote{r3}} (\sphinxstyleliteralemphasis{\sphinxupquote{float}}) \textendash{} \(r_3\).

\item {} 
\sphinxstyleliteralstrong{\sphinxupquote{s}} (\sphinxstyleliteralemphasis{\sphinxupquote{float}}) \textendash{} the sign \(s\) selecting between the two solutions
for the second quasiparticle’s momentum.

\end{itemize}

\item[{Returns}] \leavevmode
differential probability.

\item[{Return type}] \leavevmode
float

\item[{Raises}] \leavevmode
\sphinxstyleliteralstrong{\sphinxupquote{ValueError}} \textendash{} if \(r_q = 0\).

\end{description}\end{quote}

\end{fulllineitems}

\index{q\_rate() (scdc.initial.distribution.integral.RateIntegrator method)@\spxentry{q\_rate()}\spxextra{scdc.initial.distribution.integral.RateIntegrator method}}

\begin{fulllineitems}
\phantomsection\label{\detokenize{code_structure:scdc.initial.distribution.integral.RateIntegrator.q_rate}}\pysiglinewithargsret{\sphinxbfcode{\sphinxupquote{q\_rate}}}{\emph{\DUrole{n}{r1}}, \emph{\DUrole{n}{rq}}, \emph{\DUrole{n}{cq}}, \emph{\DUrole{n}{order}\DUrole{o}{=}\DUrole{default_value}{None}}}{}
Integrated rate at fixed \(r_1\) and \(q\).
\begin{quote}\begin{description}
\item[{Parameters}] \leavevmode\begin{itemize}
\item {} 
\sphinxstyleliteralstrong{\sphinxupquote{r1}} (\sphinxstyleliteralemphasis{\sphinxupquote{float}}) \textendash{} \(r_1\).

\item {} 
\sphinxstyleliteralstrong{\sphinxupquote{rq}} (\sphinxstyleliteralemphasis{\sphinxupquote{float}}) \textendash{} \(r_q\).

\item {} 
\sphinxstyleliteralstrong{\sphinxupquote{cq}} (\sphinxstyleliteralemphasis{\sphinxupquote{float}}) \textendash{} \(c_q\).

\end{itemize}

\item[{Returns}] \leavevmode
differential rate (non\sphinxhyphen{}normalized).

\item[{Return type}] \leavevmode
float

\end{description}\end{quote}

\end{fulllineitems}

\index{r3\_domain() (scdc.initial.distribution.integral.RateIntegrator method)@\spxentry{r3\_domain()}\spxextra{scdc.initial.distribution.integral.RateIntegrator method}}

\begin{fulllineitems}
\phantomsection\label{\detokenize{code_structure:scdc.initial.distribution.integral.RateIntegrator.r3_domain}}\pysiglinewithargsret{\sphinxbfcode{\sphinxupquote{r3\_domain}}}{\emph{\DUrole{n}{r1}}, \emph{\DUrole{n}{rq}}, \emph{\DUrole{n}{cq}}}{}
Compute the minimum and maximum allowed values of \(r_3\).
\begin{quote}\begin{description}
\item[{Parameters}] \leavevmode\begin{itemize}
\item {} 
\sphinxstyleliteralstrong{\sphinxupquote{r1}} (\sphinxstyleliteralemphasis{\sphinxupquote{float}}) \textendash{} \(r_1\).

\item {} 
\sphinxstyleliteralstrong{\sphinxupquote{rq}} (\sphinxstyleliteralemphasis{\sphinxupquote{float}}) \textendash{} \(r_q\).

\item {} 
\sphinxstyleliteralstrong{\sphinxupquote{cq}} (\sphinxstyleliteralemphasis{\sphinxupquote{float}}) \textendash{} \(c_q\).

\end{itemize}

\item[{Returns}] \leavevmode
lower bound on \(r_3\).
float: upper bound on \(r_3\).

\item[{Return type}] \leavevmode
float

\end{description}\end{quote}

\end{fulllineitems}


\end{fulllineitems}



\section{Analysis tools}
\label{\detokenize{analysis:analysis-tools}}\label{\detokenize{analysis::doc}}

\subsection{Analysis functions}
\label{\detokenize{analysis:module-scdc.analyze}}\label{\detokenize{analysis:analysis-functions}}\index{module@\spxentry{module}!scdc.analyze@\spxentry{scdc.analyze}}\index{scdc.analyze@\spxentry{scdc.analyze}!module@\spxentry{module}}
This module defines functions for analysis of simulated ensembles.

Many of these are ‘old’, in that they were written at one time for a type of
analysis that has not been used since. However, they are preserved here for
future applications.
\index{norm\_asymmetry() (in module scdc.analyze)@\spxentry{norm\_asymmetry()}\spxextra{in module scdc.analyze}}

\begin{fulllineitems}
\phantomsection\label{\detokenize{analysis:scdc.analyze.norm_asymmetry}}\pysiglinewithargsret{\sphinxcode{\sphinxupquote{scdc.analyze.}}\sphinxbfcode{\sphinxupquote{norm\_asymmetry}}}{\emph{angles}, \emph{distance\_function=\textless{}function p\_dist.\textless{}locals\textgreater{}.\_dist\textgreater{}}, \emph{n\_bins=50}}{}
Find the asymmetry as the distance from an isotropic distribution.

For any norm, this will give zero for perfectly isotropic scattering. The
maximum depends on the norm. For the default L\textasciicircum{}1 norm, if the scattering is
purely directional, corresponding to a delta function, the difference will
be 0.5 over the whole interval and then the delta function integrates to 1,
so that’ll be a maximum of 2. For the L\textasciicircum{}2 norm, the norm of the delta is
not well defined.

** However, because of the default L\textasciicircum{}1 behavior, the result is divided by
2 regardless of the distance function. **

Because we’re working with a histogram anyway, we don’t want to use any
actual quadrature. Instead, we want to use a p\sphinxhyphen{}norm of some kind. So the
distance function here is not really a norm, but a single\sphinxhyphen{}point integrand
thereof. For example, to use an L\textasciicircum{}2 norm, the distance function should be
\begin{quote}

lambda x: np.abs(x)**2
\end{quote}

and the rest will be taken care of internally.
\begin{quote}\begin{description}
\item[{Parameters}] \leavevmode\begin{itemize}
\item {} 
\sphinxstyleliteralstrong{\sphinxupquote{angles}} (\sphinxcode{\sphinxupquote{ndarray}}) \textendash{} cos(theta) values. A \sphinxtitleref{ParticleCollection}
object can be provided instead.

\item {} 
\sphinxstyleliteralstrong{\sphinxupquote{distance\_function}} (\sphinxstyleliteralemphasis{\sphinxupquote{function}}\sphinxstyleliteralemphasis{\sphinxupquote{, }}\sphinxstyleliteralemphasis{\sphinxupquote{optional}}) \textendash{} a function of one variable
giving the integrand of the norm. Defaults to \sphinxtitleref{p\_dist(1)}.

\item {} 
\sphinxstyleliteralstrong{\sphinxupquote{n\_bins}} (\sphinxstyleliteralemphasis{\sphinxupquote{int}}\sphinxstyleliteralemphasis{\sphinxupquote{, }}\sphinxstyleliteralemphasis{\sphinxupquote{optional}}) \textendash{} number of bins to use for the norm. Defaults
to 50.

\end{itemize}

\item[{Returns}] \leavevmode
norm/2 of the distance from an isotropic distribution.

\item[{Return type}] \leavevmode
float

\end{description}\end{quote}

\end{fulllineitems}

\index{p\_dist() (in module scdc.analyze)@\spxentry{p\_dist()}\spxextra{in module scdc.analyze}}

\begin{fulllineitems}
\phantomsection\label{\detokenize{analysis:scdc.analyze.p_dist}}\pysiglinewithargsret{\sphinxcode{\sphinxupquote{scdc.analyze.}}\sphinxbfcode{\sphinxupquote{p\_dist}}}{\emph{\DUrole{n}{p}}}{}
Factory for L\textasciicircum{}p\sphinxhyphen{}norm integrands.
\begin{quote}\begin{description}
\item[{Parameters}] \leavevmode
\sphinxstyleliteralstrong{\sphinxupquote{p}} (\sphinxstyleliteralemphasis{\sphinxupquote{float}}) \textendash{} p for the L\textasciicircum{}p norm.

\item[{Returns}] \leavevmode
L\textasciicircum{}p norm integrand as a function of one argument.

\item[{Return type}] \leavevmode
function

\end{description}\end{quote}

\end{fulllineitems}

\index{plane\_asymmetry() (in module scdc.analyze)@\spxentry{plane\_asymmetry()}\spxextra{in module scdc.analyze}}

\begin{fulllineitems}
\phantomsection\label{\detokenize{analysis:scdc.analyze.plane_asymmetry}}\pysiglinewithargsret{\sphinxcode{\sphinxupquote{scdc.analyze.}}\sphinxbfcode{\sphinxupquote{plane\_asymmetry}}}{\emph{\DUrole{n}{angles}}, \emph{\DUrole{n}{n\_bins}\DUrole{o}{=}\DUrole{default_value}{100}}, \emph{\DUrole{n}{width}\DUrole{o}{=}\DUrole{default_value}{1}}}{}
Find the asymmetry in a sliding bin of fixed width in cos(theta).

For cos(theta) = 1, this corresponds to forward\sphinxhyphen{}backward asymmetry.
\begin{quote}\begin{description}
\item[{Parameters}] \leavevmode\begin{itemize}
\item {} 
\sphinxstyleliteralstrong{\sphinxupquote{angles}} (\sphinxcode{\sphinxupquote{ndarray}}) \textendash{} cos(theta) values. A \sphinxtitleref{ParticleCollection}
object can be provided instead.

\item {} 
\sphinxstyleliteralstrong{\sphinxupquote{width}} (\sphinxstyleliteralemphasis{\sphinxupquote{float}}\sphinxstyleliteralemphasis{\sphinxupquote{, }}\sphinxstyleliteralemphasis{\sphinxupquote{optional}}) \textendash{} width of the sliding bin. Defaults to 1.

\end{itemize}

\end{description}\end{quote}

\end{fulllineitems}

\index{qp\_angle\_pairs() (in module scdc.analyze)@\spxentry{qp\_angle\_pairs()}\spxextra{in module scdc.analyze}}

\begin{fulllineitems}
\phantomsection\label{\detokenize{analysis:scdc.analyze.qp_angle_pairs}}\pysiglinewithargsret{\sphinxcode{\sphinxupquote{scdc.analyze.}}\sphinxbfcode{\sphinxupquote{qp\_angle\_pairs}}}{\emph{\DUrole{n}{event}}}{}
Final\sphinxhyphen{}state QP angles in canonical pairs.

Here ‘canonical’ pairing means the following. The number of quasiparticles
produced in any event must be even, so we sort them by energy and then
divide into a low\sphinxhyphen{}energy half and a high\sphinxhyphen{}energy half. The lowest low\sphinxhyphen{}energy
QP is paired with the lowest high\sphinxhyphen{}energy QP, the second\sphinxhyphen{}lowest with the
second\sphinxhyphen{}lowest, and so on. There is nothing important about this order
except that it is well\sphinxhyphen{}defined.
\begin{quote}\begin{description}
\item[{Parameters}] \leavevmode
\sphinxstyleliteralstrong{\sphinxupquote{event}} ({\hyperref[\detokenize{code_structure:scdc.event.Event}]{\sphinxcrossref{\sphinxstyleliteralemphasis{\sphinxupquote{Event}}}}}) \textendash{} the event for which to find final\sphinxhyphen{}state QP pairs.

\item[{Returns}] \leavevmode
a 2d array in which each row is a pair.

\item[{Return type}] \leavevmode
ndarray

\end{description}\end{quote}

\end{fulllineitems}



\subsection{Plotting functions}
\label{\detokenize{analysis:module-scdc.plot}}\label{\detokenize{analysis:plotting-functions}}\index{module@\spxentry{module}!scdc.plot@\spxentry{scdc.plot}}\index{scdc.plot@\spxentry{scdc.plot}!module@\spxentry{module}}
This module defines plotting styles and routines. Some are out of date
but are retained for possible future use.
\index{latex\_exp\_format() (in module scdc.plot)@\spxentry{latex\_exp\_format()}\spxextra{in module scdc.plot}}

\begin{fulllineitems}
\phantomsection\label{\detokenize{analysis:scdc.plot.latex_exp_format}}\pysiglinewithargsret{\sphinxcode{\sphinxupquote{scdc.plot.}}\sphinxbfcode{\sphinxupquote{latex\_exp\_format}}}{\emph{\DUrole{n}{x}}, \emph{\DUrole{n}{mfmt}\DUrole{o}{=}\DUrole{default_value}{\textquotesingle{}\%.3f\textquotesingle{}}}}{}
Format a number in scientific notation in LaTeX.
\begin{quote}\begin{description}
\item[{Parameters}] \leavevmode\begin{itemize}
\item {} 
\sphinxstyleliteralstrong{\sphinxupquote{x}} (\sphinxstyleliteralemphasis{\sphinxupquote{float}}) \textendash{} number to format.

\item {} 
\sphinxstyleliteralstrong{\sphinxupquote{mfmt}} (\sphinxstyleliteralemphasis{\sphinxupquote{str}}\sphinxstyleliteralemphasis{\sphinxupquote{, }}\sphinxstyleliteralemphasis{\sphinxupquote{optional}}) \textendash{} format string for mantissa. Defaults to ‘\%.3f’.

\end{itemize}

\item[{Returns}] \leavevmode
LaTeX string (no \$’s).

\item[{Return type}] \leavevmode
str

\end{description}\end{quote}

\end{fulllineitems}

\index{tree\_plot() (in module scdc.plot)@\spxentry{tree\_plot()}\spxextra{in module scdc.plot}}

\begin{fulllineitems}
\phantomsection\label{\detokenize{analysis:scdc.plot.tree_plot}}\pysiglinewithargsret{\sphinxcode{\sphinxupquote{scdc.plot.}}\sphinxbfcode{\sphinxupquote{tree\_plot}}}{\emph{\DUrole{n}{particle}}, \emph{\DUrole{n}{origin}\DUrole{o}{=}\DUrole{default_value}{(0, 0)}}, \emph{\DUrole{o}{**}\DUrole{n}{kwargs}}}{}
Plot all child scattering events.
\begin{quote}\begin{description}
\item[{Parameters}] \leavevmode\begin{itemize}
\item {} 
\sphinxstyleliteralstrong{\sphinxupquote{fig}} (\sphinxcode{\sphinxupquote{Figure}}, optional) \textendash{} matplotlib figure object.

\item {} 
\sphinxstyleliteralstrong{\sphinxupquote{ax}} (\sphinxcode{\sphinxupquote{AxesSubplot}}, optional) \textendash{} matplotlib axis object.

\item {} 
\sphinxstyleliteralstrong{\sphinxupquote{origin}} (\sphinxcode{\sphinxupquote{tuple}} of \sphinxcode{\sphinxupquote{float}}, optional) \textendash{} starting
coordinates for the tree.

\item {} 
\sphinxstyleliteralstrong{\sphinxupquote{dm\_color}} (\sphinxstyleliteralemphasis{\sphinxupquote{str}}\sphinxstyleliteralemphasis{\sphinxupquote{, }}\sphinxstyleliteralemphasis{\sphinxupquote{optional}}) \textendash{} color for DM lines.

\item {} 
\sphinxstyleliteralstrong{\sphinxupquote{phonon\_color}} (\sphinxstyleliteralemphasis{\sphinxupquote{str}}\sphinxstyleliteralemphasis{\sphinxupquote{, }}\sphinxstyleliteralemphasis{\sphinxupquote{optional}}) \textendash{} color for phonon lines.

\item {} 
\sphinxstyleliteralstrong{\sphinxupquote{qp\_color}} (\sphinxstyleliteralemphasis{\sphinxupquote{str}}\sphinxstyleliteralemphasis{\sphinxupquote{, }}\sphinxstyleliteralemphasis{\sphinxupquote{optional}}) \textendash{} color for quasiparticle lines.

\item {} 
\sphinxstyleliteralstrong{\sphinxupquote{min\_linewidth}} (\sphinxstyleliteralemphasis{\sphinxupquote{float}}\sphinxstyleliteralemphasis{\sphinxupquote{, }}\sphinxstyleliteralemphasis{\sphinxupquote{optional}}) \textendash{} smallest linewidth (E = Delta).

\item {} 
\sphinxstyleliteralstrong{\sphinxupquote{max\_linewidth}} (\sphinxstyleliteralemphasis{\sphinxupquote{float}}\sphinxstyleliteralemphasis{\sphinxupquote{, }}\sphinxstyleliteralemphasis{\sphinxupquote{optional}}) \textendash{} largest linewidth, corresponding
to the energy of the initial excitation.

\item {} 
\sphinxstyleliteralstrong{\sphinxupquote{max\_linewidth\_energy}} (\sphinxstyleliteralemphasis{\sphinxupquote{float}}\sphinxstyleliteralemphasis{\sphinxupquote{, }}\sphinxstyleliteralemphasis{\sphinxupquote{optional}}) \textendash{} energy for max linewidth.

\item {} 
\sphinxstyleliteralstrong{\sphinxupquote{final\_distance}} (\sphinxstyleliteralemphasis{\sphinxupquote{float}}\sphinxstyleliteralemphasis{\sphinxupquote{, }}\sphinxstyleliteralemphasis{\sphinxupquote{optional}}) \textendash{} if specified, final (ballistic)
state lines will be extended to this distance from (0, 0).

\item {} 
\sphinxstyleliteralstrong{\sphinxupquote{alpha}} (\sphinxstyleliteralemphasis{\sphinxupquote{float}}\sphinxstyleliteralemphasis{\sphinxupquote{, }}\sphinxstyleliteralemphasis{\sphinxupquote{optional}}) \textendash{} opacity.

\end{itemize}

\end{description}\end{quote}

\end{fulllineitems}



\section{Using the code}
\label{\detokenize{using:using-the-code}}\label{\detokenize{using::doc}}

\subsection{Contents}
\label{\detokenize{using:contents}}

\subsubsection{Batch interface}
\label{\detokenize{interface:batch-interface}}\label{\detokenize{interface::doc}}

\paragraph{MPI capabilities}
\label{\detokenize{interface:module-scdc.mpi.base}}\label{\detokenize{interface:mpi-capabilities}}\index{module@\spxentry{module}!scdc.mpi.base@\spxentry{scdc.mpi.base}}\index{scdc.mpi.base@\spxentry{scdc.mpi.base}!module@\spxentry{module}}
This module defines a custom MPI scatter\sphinxhyphen{}gather manager. It can be used
independently of the remainder of the code.

\phantomsection\label{\detokenize{interface:module-scdc.mpi.sim}}\index{module@\spxentry{module}!scdc.mpi.sim@\spxentry{scdc.mpi.sim}}\index{scdc.mpi.sim@\spxentry{scdc.mpi.sim}!module@\spxentry{module}}
This module enables parallelized down\sphinxhyphen{}conversion using MPI.
\index{particle\_as\_dict() (in module scdc.mpi.sim)@\spxentry{particle\_as\_dict()}\spxextra{in module scdc.mpi.sim}}

\begin{fulllineitems}
\phantomsection\label{\detokenize{interface:scdc.mpi.sim.particle_as_dict}}\pysiglinewithargsret{\sphinxcode{\sphinxupquote{scdc.mpi.sim.}}\sphinxbfcode{\sphinxupquote{particle\_as\_dict}}}{\emph{\DUrole{n}{p}}}{}
Convert a \sphinxcode{\sphinxupquote{Particle}} object to a lightweight dict.

The dictionary form has keys ‘shortname’, ‘momentum’, and ‘cos\_theta’.
\begin{quote}\begin{description}
\item[{Parameters}] \leavevmode
\sphinxstyleliteralstrong{\sphinxupquote{p}} (\sphinxcode{\sphinxupquote{Particle}}) \textendash{} particle to convert to a dict.

\item[{Returns}] \leavevmode
a simple dictionary form of the particle.

\item[{Return type}] \leavevmode
dict

\end{description}\end{quote}

\end{fulllineitems}

\index{particle\_from\_dict() (in module scdc.mpi.sim)@\spxentry{particle\_from\_dict()}\spxextra{in module scdc.mpi.sim}}

\begin{fulllineitems}
\phantomsection\label{\detokenize{interface:scdc.mpi.sim.particle_from_dict}}\pysiglinewithargsret{\sphinxcode{\sphinxupquote{scdc.mpi.sim.}}\sphinxbfcode{\sphinxupquote{particle\_from\_dict}}}{\emph{\DUrole{n}{d}}, \emph{\DUrole{n}{material}}}{}
Convert the output of \sphinxtitleref{particle\_as\_dict} back to an object.
\begin{quote}\begin{description}
\item[{Parameters}] \leavevmode\begin{itemize}
\item {} 
\sphinxstyleliteralstrong{\sphinxupquote{d}} (\sphinxstyleliteralemphasis{\sphinxupquote{dict}}) \textendash{} dictionary to convert to a particle.

\item {} 
\sphinxstyleliteralstrong{\sphinxupquote{material}} (\sphinxcode{\sphinxupquote{Material}}) \textendash{} the material to use for this particle.
Material data is not included in the dict representation.

\end{itemize}

\item[{Returns}] \leavevmode
an object form of the dict.

\item[{Return type}] \leavevmode
\sphinxcode{\sphinxupquote{Particle}}

\end{description}\end{quote}

\end{fulllineitems}

\phantomsection\label{\detokenize{interface:module-scdc.mpi.initial}}\index{module@\spxentry{module}!scdc.mpi.initial@\spxentry{scdc.mpi.initial}}\index{scdc.mpi.initial@\spxentry{scdc.mpi.initial}!module@\spxentry{module}}
This module enables parallelized initial\sphinxhyphen{}QP sampling using MPI.


\paragraph{Command\sphinxhyphen{}line interface}
\label{\detokenize{interface:module-scdc.interface}}\label{\detokenize{interface:command-line-interface}}\index{module@\spxentry{module}!scdc.interface@\spxentry{scdc.interface}}\index{scdc.interface@\spxentry{scdc.interface}!module@\spxentry{module}}
This module defines utilities for running ensembles through the command
line or batch queue.

Runs are defined by a configuration file in json format. The json file should
have the following keys:
\begin{quote}

outfile: path to the output file. Extension will be appended if missing.
copies: number of copies of the initial state to run. Defaults to 1.
initial: initial state to use.
\begin{itemize}
\item {} \begin{description}
\item[{If a string, this is taken to be the path to a file. The file should}] \leavevmode
have columns of the form
\begin{quote}

pDM  cDM  p1  c1  p2  c2
\end{quote}

where \sphinxtitleref{p} and \sphinxtitleref{c} are the momentum and cos(theta) for the DM, QP1,
and QP2, respectively.

\end{description}

\item {} \begin{description}
\item[{If a dict, this is taken to specify the parameters of an initial}] \leavevmode\begin{description}
\item[{state. In this case, the allowed keys are:}] \leavevmode\begin{itemize}
\item {} 
\sphinxtitleref{momentum}

\item {} 
\sphinxtitleref{energy}

\item {} 
\sphinxtitleref{shortname}

\item {} 
\sphinxtitleref{cos\_theta}

\end{itemize}

\end{description}

Exactly one of \sphinxtitleref{energy} and \sphinxtitleref{momentum} must be specified. Any of
these keys may be given as a list of values. In this case, one
enemble is produced for each value in the list (or for each
combination of list values, if applicable).

\end{description}

\end{itemize}
\begin{description}
\item[{material: material data to use (dict).}] \leavevmode\begin{itemize}
\item {} 
If omitted or null, the default is \sphinxtitleref{materials.Aluminum}.

\item {} \begin{description}
\item[{Allowed keys are:}] \leavevmode\begin{itemize}
\item {} 
\sphinxtitleref{gamma}

\item {} 
\sphinxtitleref{c\_s}

\item {} 
\sphinxtitleref{T\_c}

\item {} 
\sphinxtitleref{Delta}

\item {} 
\sphinxtitleref{E\_F}

\item {} 
\sphinxtitleref{m\_star}

\item {} 
\sphinxtitleref{beta}

\end{itemize}

\end{description}

\item {} 
Any keys omitted are taken from the default (\sphinxtitleref{materials.Aluminum}).

\item {} \begin{description}
\item[{As with the \sphinxtitleref{initial} key, any values specified as lists will produce}] \leavevmode
one ensemble for each value or combination of values.

\end{description}

\end{itemize}

\end{description}

params: an arbitrary dict of additional labels with primitive values.
\end{quote}
\index{Configuration (class in scdc.interface)@\spxentry{Configuration}\spxextra{class in scdc.interface}}

\begin{fulllineitems}
\phantomsection\label{\detokenize{interface:scdc.interface.Configuration}}\pysiglinewithargsret{\sphinxbfcode{\sphinxupquote{class }}\sphinxcode{\sphinxupquote{scdc.interface.}}\sphinxbfcode{\sphinxupquote{Configuration}}}{\emph{\DUrole{o}{**}\DUrole{n}{kwargs}}}{}
Configuration generator for multi\sphinxhyphen{}task downconversion runs.
\begin{quote}\begin{description}
\item[{Parameters}] \leavevmode\begin{itemize}
\item {} 
\sphinxstyleliteralstrong{\sphinxupquote{outfile}} (\sphinxstyleliteralemphasis{\sphinxupquote{str}}) \textendash{} path to the output file. Extension will be appended
if missing.

\item {} 
\sphinxstyleliteralstrong{\sphinxupquote{copies}} (\sphinxstyleliteralemphasis{\sphinxupquote{int}}\sphinxstyleliteralemphasis{\sphinxupquote{, }}\sphinxstyleliteralemphasis{\sphinxupquote{optional}}) \textendash{} number of copies to run. Defaults to 1.

\item {} 
\sphinxstyleliteralstrong{\sphinxupquote{initial}} (\sphinxstyleliteralemphasis{\sphinxupquote{object}}) \textendash{} initial particle specification.

\item {} 
\sphinxstyleliteralstrong{\sphinxupquote{n\_initial}} (\sphinxstyleliteralemphasis{\sphinxupquote{int}}\sphinxstyleliteralemphasis{\sphinxupquote{ or }}\sphinxstyleliteralemphasis{\sphinxupquote{dict}}\sphinxstyleliteralemphasis{\sphinxupquote{, }}\sphinxstyleliteralemphasis{\sphinxupquote{optional}}) \textendash{} number of initial particles to
select for downconversion. Selection is random with replacement. If
\sphinxcode{\sphinxupquote{None}}, use all initial particles. A two\sphinxhyphen{}layer dict may be
provided using the mediator mass and DM mass as keys, with an
‘other’ entry as a fallback. Defaults to \sphinxcode{\sphinxupquote{None}}.

\item {} 
\sphinxstyleliteralstrong{\sphinxupquote{material}} (\sphinxstyleliteralemphasis{\sphinxupquote{object}}) \textendash{} material specification.

\item {} 
\sphinxstyleliteralstrong{\sphinxupquote{params}} (\sphinxstyleliteralemphasis{\sphinxupquote{dict}}\sphinxstyleliteralemphasis{\sphinxupquote{, }}\sphinxstyleliteralemphasis{\sphinxupquote{optional}}) \textendash{} arbitrary parameter dict for labels.
Defaults to \sphinxtitleref{\{\}}.

\end{itemize}

\end{description}\end{quote}
\index{outfile (scdc.interface.Configuration attribute)@\spxentry{outfile}\spxextra{scdc.interface.Configuration attribute}}

\begin{fulllineitems}
\phantomsection\label{\detokenize{interface:scdc.interface.Configuration.outfile}}\pysigline{\sphinxbfcode{\sphinxupquote{outfile}}}
original outfile specification.
\begin{quote}\begin{description}
\item[{Type}] \leavevmode
str

\end{description}\end{quote}

\end{fulllineitems}

\index{copies (scdc.interface.Configuration attribute)@\spxentry{copies}\spxextra{scdc.interface.Configuration attribute}}

\begin{fulllineitems}
\phantomsection\label{\detokenize{interface:scdc.interface.Configuration.copies}}\pysigline{\sphinxbfcode{\sphinxupquote{copies}}}
original copies specification.
\begin{quote}\begin{description}
\item[{Type}] \leavevmode
int

\end{description}\end{quote}

\end{fulllineitems}

\index{material (scdc.interface.Configuration attribute)@\spxentry{material}\spxextra{scdc.interface.Configuration attribute}}

\begin{fulllineitems}
\phantomsection\label{\detokenize{interface:scdc.interface.Configuration.material}}\pysigline{\sphinxbfcode{\sphinxupquote{material}}}
original material specification.
\begin{quote}\begin{description}
\item[{Type}] \leavevmode
object

\end{description}\end{quote}

\end{fulllineitems}

\index{params (scdc.interface.Configuration attribute)@\spxentry{params}\spxextra{scdc.interface.Configuration attribute}}

\begin{fulllineitems}
\phantomsection\label{\detokenize{interface:scdc.interface.Configuration.params}}\pysigline{\sphinxbfcode{\sphinxupquote{params}}}
original params specification.
\begin{quote}\begin{description}
\item[{Type}] \leavevmode
dict

\end{description}\end{quote}

\end{fulllineitems}

\index{materials (scdc.interface.Configuration attribute)@\spxentry{materials}\spxextra{scdc.interface.Configuration attribute}}

\begin{fulllineitems}
\phantomsection\label{\detokenize{interface:scdc.interface.Configuration.materials}}\pysigline{\sphinxbfcode{\sphinxupquote{materials}}}
material specifications.
\begin{quote}\begin{description}
\item[{Type}] \leavevmode
object

\end{description}\end{quote}

\end{fulllineitems}

\index{ensemble\_tasks (scdc.interface.Configuration attribute)@\spxentry{ensemble\_tasks}\spxextra{scdc.interface.Configuration attribute}}

\begin{fulllineitems}
\phantomsection\label{\detokenize{interface:scdc.interface.Configuration.ensemble_tasks}}\pysigline{\sphinxbfcode{\sphinxupquote{ensemble\_tasks}}}
one task per
ensemble.
\begin{quote}\begin{description}
\item[{Type}] \leavevmode
\sphinxcode{\sphinxupquote{list}} of {\hyperref[\detokenize{interface:scdc.interface.EnsembleTask}]{\sphinxcrossref{\sphinxcode{\sphinxupquote{EnsembleTask}}}}}

\end{description}\end{quote}

\end{fulllineitems}

\index{task\_by\_id (scdc.interface.Configuration attribute)@\spxentry{task\_by\_id}\spxextra{scdc.interface.Configuration attribute}}

\begin{fulllineitems}
\phantomsection\label{\detokenize{interface:scdc.interface.Configuration.task_by_id}}\pysigline{\sphinxbfcode{\sphinxupquote{task\_by\_id}}}
a dictionary mapping ensemble task IDs to
\sphinxtitleref{EnsembleTask} objects.
\begin{quote}\begin{description}
\item[{Type}] \leavevmode
dict

\end{description}\end{quote}

\end{fulllineitems}

\index{save() (scdc.interface.Configuration method)@\spxentry{save()}\spxextra{scdc.interface.Configuration method}}

\begin{fulllineitems}
\phantomsection\label{\detokenize{interface:scdc.interface.Configuration.save}}\pysiglinewithargsret{\sphinxbfcode{\sphinxupquote{save}}}{}{}
Save an HDF5 representation of this run and the output.

\end{fulllineitems}


\end{fulllineitems}

\index{EnsembleTask (class in scdc.interface)@\spxentry{EnsembleTask}\spxextra{class in scdc.interface}}

\begin{fulllineitems}
\phantomsection\label{\detokenize{interface:scdc.interface.EnsembleTask}}\pysiglinewithargsret{\sphinxbfcode{\sphinxupquote{class }}\sphinxcode{\sphinxupquote{scdc.interface.}}\sphinxbfcode{\sphinxupquote{EnsembleTask}}}{\emph{\DUrole{o}{**}\DUrole{n}{kwargs}}}{}
Container class for single ensembles.
\begin{quote}\begin{description}
\item[{Parameters}] \leavevmode\begin{itemize}
\item {} 
\sphinxstyleliteralstrong{\sphinxupquote{initial}} (\sphinxcode{\sphinxupquote{list}} of \sphinxcode{\sphinxupquote{Event}}) \textendash{} initial events.

\item {} 
\sphinxstyleliteralstrong{\sphinxupquote{material}} (\sphinxcode{\sphinxupquote{Material}}) \textendash{} material object.

\item {} 
\sphinxstyleliteralstrong{\sphinxupquote{params}} (\sphinxstyleliteralemphasis{\sphinxupquote{dict}}\sphinxstyleliteralemphasis{\sphinxupquote{, }}\sphinxstyleliteralemphasis{\sphinxupquote{optional}}) \textendash{} arbitrary parameter dict for labels.
Defaults to \sphinxtitleref{\{\}}.

\item {} 
\sphinxstyleliteralstrong{\sphinxupquote{task\_id}} (\sphinxstyleliteralemphasis{\sphinxupquote{object}}\sphinxstyleliteralemphasis{\sphinxupquote{, }}\sphinxstyleliteralemphasis{\sphinxupquote{optional}}) \textendash{} a label for this task. Defaults to \sphinxtitleref{None}.

\end{itemize}

\end{description}\end{quote}
\index{initial (scdc.interface.EnsembleTask attribute)@\spxentry{initial}\spxextra{scdc.interface.EnsembleTask attribute}}

\begin{fulllineitems}
\phantomsection\label{\detokenize{interface:scdc.interface.EnsembleTask.initial}}\pysigline{\sphinxbfcode{\sphinxupquote{initial}}}
initial events.
\begin{quote}\begin{description}
\item[{Type}] \leavevmode
\sphinxcode{\sphinxupquote{list}} of \sphinxcode{\sphinxupquote{Event}}

\end{description}\end{quote}

\end{fulllineitems}

\index{material (scdc.interface.EnsembleTask attribute)@\spxentry{material}\spxextra{scdc.interface.EnsembleTask attribute}}

\begin{fulllineitems}
\phantomsection\label{\detokenize{interface:scdc.interface.EnsembleTask.material}}\pysigline{\sphinxbfcode{\sphinxupquote{material}}}
material object.
\begin{quote}\begin{description}
\item[{Type}] \leavevmode
\sphinxcode{\sphinxupquote{Material}}

\end{description}\end{quote}

\end{fulllineitems}

\index{params (scdc.interface.EnsembleTask attribute)@\spxentry{params}\spxextra{scdc.interface.EnsembleTask attribute}}

\begin{fulllineitems}
\phantomsection\label{\detokenize{interface:scdc.interface.EnsembleTask.params}}\pysigline{\sphinxbfcode{\sphinxupquote{params}}}
arbitrary parameter dict for labels.
\begin{quote}\begin{description}
\item[{Type}] \leavevmode
dict

\end{description}\end{quote}

\end{fulllineitems}

\index{task\_id (scdc.interface.EnsembleTask attribute)@\spxentry{task\_id}\spxextra{scdc.interface.EnsembleTask attribute}}

\begin{fulllineitems}
\phantomsection\label{\detokenize{interface:scdc.interface.EnsembleTask.task_id}}\pysigline{\sphinxbfcode{\sphinxupquote{task\_id}}}
a label for this task.
\begin{quote}\begin{description}
\item[{Type}] \leavevmode
object

\end{description}\end{quote}

\end{fulllineitems}

\index{result (scdc.interface.EnsembleTask attribute)@\spxentry{result}\spxextra{scdc.interface.EnsembleTask attribute}}

\begin{fulllineitems}
\phantomsection\label{\detokenize{interface:scdc.interface.EnsembleTask.result}}\pysigline{\sphinxbfcode{\sphinxupquote{result}}}
results of running this task.
\begin{quote}\begin{description}
\item[{Type}] \leavevmode
\sphinxcode{\sphinxupquote{ndarray}}

\end{description}\end{quote}

\end{fulllineitems}


\end{fulllineitems}

\index{InitialConfiguration (class in scdc.interface)@\spxentry{InitialConfiguration}\spxextra{class in scdc.interface}}

\begin{fulllineitems}
\phantomsection\label{\detokenize{interface:scdc.interface.InitialConfiguration}}\pysiglinewithargsret{\sphinxbfcode{\sphinxupquote{class }}\sphinxcode{\sphinxupquote{scdc.interface.}}\sphinxbfcode{\sphinxupquote{InitialConfiguration}}}{\emph{\DUrole{o}{**}\DUrole{n}{kwargs}}}{}
Configuration generator for multi\sphinxhyphen{}task initial\sphinxhyphen{}particle runs.

At present, it is assumed that only the DM and mediator mass vary. The
velocity is fixed to 1e\sphinxhyphen{}3.
\index{save() (scdc.interface.InitialConfiguration method)@\spxentry{save()}\spxextra{scdc.interface.InitialConfiguration method}}

\begin{fulllineitems}
\phantomsection\label{\detokenize{interface:scdc.interface.InitialConfiguration.save}}\pysiglinewithargsret{\sphinxbfcode{\sphinxupquote{save}}}{}{}
Save an HDF5 representation of this run and the output.

\end{fulllineitems}


\end{fulllineitems}

\index{InitialTask (class in scdc.interface)@\spxentry{InitialTask}\spxextra{class in scdc.interface}}

\begin{fulllineitems}
\phantomsection\label{\detokenize{interface:scdc.interface.InitialTask}}\pysiglinewithargsret{\sphinxbfcode{\sphinxupquote{class }}\sphinxcode{\sphinxupquote{scdc.interface.}}\sphinxbfcode{\sphinxupquote{InitialTask}}}{\emph{\DUrole{o}{*}\DUrole{n}{args}}, \emph{\DUrole{o}{**}\DUrole{n}{kwargs}}}{}
Container class for single initial\sphinxhyphen{}ensemble generation tasks.

At present, it is assumed that only the DM and mediator mass vary. The
velocity is fixed to 1e\sphinxhyphen{}3.

\end{fulllineitems}

\index{expand\_dict() (in module scdc.interface)@\spxentry{expand\_dict()}\spxextra{in module scdc.interface}}

\begin{fulllineitems}
\phantomsection\label{\detokenize{interface:scdc.interface.expand_dict}}\pysiglinewithargsret{\sphinxcode{\sphinxupquote{scdc.interface.}}\sphinxbfcode{\sphinxupquote{expand\_dict}}}{\emph{\DUrole{n}{d}}}{}
Expand a dict of lists to a list of dicts.

The idea is to take a dict for which some values are iterable, and convert
this to a single list of dicts, each of which has no listlike values.
\begin{quote}\begin{description}
\item[{Parameters}] \leavevmode
\sphinxstyleliteralstrong{\sphinxupquote{d}} (\sphinxstyleliteralemphasis{\sphinxupquote{dict}}) \textendash{} dict to expand.

\item[{Returns}] \leavevmode
expanded list of dicts.

\item[{Return type}] \leavevmode
\sphinxcode{\sphinxupquote{list}} of \sphinxcode{\sphinxupquote{dict}}

\end{description}\end{quote}

\end{fulllineitems}



\chapter{Indices and tables}
\label{\detokenize{index:indices-and-tables}}\begin{itemize}
\item {} 
\DUrole{xref,std,std-ref}{genindex}

\item {} 
\DUrole{xref,std,std-ref}{modindex}

\item {} 
\DUrole{xref,std,std-ref}{search}

\end{itemize}


\renewcommand{\indexname}{Python Module Index}
\begin{sphinxtheindex}
\let\bigletter\sphinxstyleindexlettergroup
\bigletter{s}
\item\relax\sphinxstyleindexentry{scdc.analyze}\sphinxstyleindexpageref{analysis:\detokenize{module-scdc.analyze}}
\item\relax\sphinxstyleindexentry{scdc.event}\sphinxstyleindexpageref{code_structure:\detokenize{module-scdc.event}}
\item\relax\sphinxstyleindexentry{scdc.initial.distribution.integral}\sphinxstyleindexpageref{code_structure:\detokenize{module-scdc.initial.distribution.integral}}
\item\relax\sphinxstyleindexentry{scdc.interaction}\sphinxstyleindexpageref{code_structure:\detokenize{module-scdc.interaction}}
\item\relax\sphinxstyleindexentry{scdc.interface}\sphinxstyleindexpageref{interface:\detokenize{module-scdc.interface}}
\item\relax\sphinxstyleindexentry{scdc.material}\sphinxstyleindexpageref{code_structure:\detokenize{module-scdc.material}}
\item\relax\sphinxstyleindexentry{scdc.mpi.base}\sphinxstyleindexpageref{interface:\detokenize{module-scdc.mpi.base}}
\item\relax\sphinxstyleindexentry{scdc.mpi.initial}\sphinxstyleindexpageref{interface:\detokenize{module-scdc.mpi.initial}}
\item\relax\sphinxstyleindexentry{scdc.mpi.sim}\sphinxstyleindexpageref{interface:\detokenize{module-scdc.mpi.sim}}
\item\relax\sphinxstyleindexentry{scdc.particle}\sphinxstyleindexpageref{code_structure:\detokenize{module-scdc.particle}}
\item\relax\sphinxstyleindexentry{scdc.plot}\sphinxstyleindexpageref{analysis:\detokenize{module-scdc.plot}}
\end{sphinxtheindex}

\renewcommand{\indexname}{Index}
\printindex
\end{document}